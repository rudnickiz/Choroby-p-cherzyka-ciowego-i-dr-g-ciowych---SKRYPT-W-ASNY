
\chapter{Zaburzenia czynnościowe pęcherzyka żółciowego i dysfunkcja zwieracza Oddiego}

\section{Definicja i obraz kliniczny}

Choroby czynnościowe pęcherzyka żółciowego i zwieracza Oddiego można rozpoznać w przypadku występowania bólu żółciowego lub nawracającego ostrego zapalenia trzustki (OZT) u chorych bez uchwytnej choroby organicznej mogącej być ich przyczyną.

Ból żółciowy to wg wytycznych rzymskich IV ból zlokalizowany w nadbrzuszu i/lub w prawym górnym kwadrancie brzucha, który spełnia wszystkie kryteria:

\begin{enumerate}
    \item osiąga stałe natężenie i trwa $\geq$30 minut

    \item pojawia się w różnych odstępach czasu (nie codziennie)

    \item jest na tyle silny, że zaburza normalną aktywność lub prowadzi do wizyt na oddziale ratunkowym

    \item nie jest istotnie ($<$20\% epizodów bólu) związany z defekacją

    \item nie jest istotnie łagodzony przez zmiany pozycji ciała ani przez leki zobojętniające lub hamujące wydzielanie kwasu solnego.
\end{enumerate}

\textbf{Bólowi żółciowemu mogą towarzyszyć:} nudności i wymioty, promieniowanie do pleców i/lub do prawej okolicy podłopatkowej, budzenie się chorego ze snu.

Ból żółciowy u chorych z pęcherzykiem żółciowym, u których nie stwierdza się choroby organicznej, może wskazywać na zaburzenie czynnościowe pęcherzyka żółciowego, natomiast u chorych po cholecystektomii można rozpoznać czynnościową dysfunkcję żółciowego zwieracza Oddiego.

Nawracające idiopatyczne OZT może wskazywać na czynnościową dysfunkcję trzustkowego zwieracza Oddiego.

\section{Rozpoznanie}

\subsection{Kryteria rozpoznania}

\begin{enumerate}
    \item Zaburzenie czynnościowe pęcherzyka żółciowego

\paragraph{Kryteria wymagane:}

\begin{enumerate}
    \item ból żółciowy

    \item niewystępowanie kamieni żółciowych (też tzw. błotka żółciowego) ani innych nieprawidłowości organicznych.
\end{enumerate}

\paragraph{Kryteria dodatkowe (nie muszą być spełnione):}

\begin{enumerate}
    \item zmniejszona frakcja wyrzutowa w scyntygrafii pęcherzyka żółciowego

    \item prawidłowa aktywność / stężenie we krwi enzymów wątrobowych, bilirubiny związanej i amylazy/lipazy (jeśli aktywność enzymów wątrobowych jest zwiększona z innej znanej przyczyny, np. niealkoholowej stłuszczeniowej choroby wątroby, to nie wyklucza zaburzenia czynnościowego pęcherzyka żółciowego).
\end{enumerate}

\item Czynnościowa dysfunkcja żółciowego zwieracza Oddiego

\paragraph{Kryteria wymagane:}

\begin{enumerate}
    \item ból żółciowy

    \item wzrost aktywności enzymów wątrobowych albo poszerzenie dróg żółciowych (nie oba objawy jednocześnie)

    \item niewystępowanie kamicy przewodowej i innych nieprawidłowości organicznych.
\end{enumerate}

\paragraph{Kryteria dodatkowe (nie muszą być spełnione):}

\begin{enumerate}
    \item prawidłowa aktywność amylazy/lipazy (może jednak być zwiększona przy niektórych napadach bólu)

    \item nieprawidłowy wynik manometrii zwieracza Oddiego (wykonuje się tylko u niektórych chorych)

    \item scyntygrafia wątroby i dróg żółciowych – przydatność wątpliwa; nie powinna wskazywać na niedrożność dróg żółciowych, często wykazuje zwolnione wydzielanie radionuklidu do żółci i/lub zwolniony przepływ żółci z wnęki wątroby do dwunastnicy.
\end{enumerate}

\item Czynnościowa dysfunkcja trzustkowego zwieracza Oddiego

\paragraph{Kryteria wymagane:}

\begin{enumerate}
    \item udokumentowane nawracające epizody OZT (typowy ból z $>$3-krotnym wzrostem aktywności amylazy lub lipazy w stosunku do normy i/lub cechy OZT w badaniach obrazowych)

    \item wykluczenie OZT o innej etiologii

    \item prawidłowy obraz w EUS

    \item nieprawidłowy wynik manometrii zwieracza.
\end{enumerate}
\end{enumerate}

\subsection{Rozpoznanie różnicowe}

Inne przyczyny bólu brzucha, przede wszystkim kamica żółciowa, choroba wrzodowa, przewlekłe zapalenie trzustki, dyspepsja czynnościowa, zespół jelita drażliwego, powikłania pooperacyjne.

Wstępna diagnostyka powinna obejmować badania biochemiczne wątroby i trzustki, gastroskopię i badania obrazowe jamy brzusznej – najlepiej EUS i MRCP. 

EUS\index{EUS} jest najdokładniejszym badaniem nieinwazyjnym pozwalającym wykluczyć kamicę przewodową i patologię brodawki Vatera. 
Nie wykonuj ECPW, jeśli nie ma wyraźnych obiektywnych cech zastoju żółci.
