
\chapter{Pierwotne stwardniające zapalenie dróg żółciowych (PSC)\index{PSC} wg Szczeklika}

\section{\DiEtp}

Przewlekła, cholestatyczna choroba wątroby o złożonej etiologii, prowadząca do uszkodzenia wewnątrzwątrobowych i zewnątrzwątrobowych dróg żółciowych.


\section{\ObriP}

U 15–45\% chorych przebieg bezobjawowy. 

\textbf{Objawy:} przewlekłe zmęczenie, świąd skóry, utrata masy ciała, objawy nawracającego zapalenia dróg żółciowych (epizody żółtaczki, gorączki i bólu w okolicy podżebrowej prawej).

W badaniu przedmiotowym zażółcenie skóry i błon śluzowych, przeczosy. 
W stadium zaawansowanym mogą wystąpić objawy marskości wątroby i jej powikłań. 
U ~70\% chorych współistnieje wrzodziejące zapalenie jelita grubego (WZJG) lub – rzadziej – choroba Leśniowskiego i Crohna (ChLC); także zapalenie trzustki (10–25\%), cukrzyca (5–15\%), inne choroby autoimmunologiczne. 
W ciągu 10–15 lat od rozpoznania $>$50\% chorych wymaga przeszczepienia wątroby.

\section{Rozpoznanie}

\subsection{Badania pomocnicze}

\begin{enumerate}
    \item \textbf{Badania laboratoryjne:}
    
    \begin{enumerate}
        \item badania biochemiczne krwi – zwiększona aktywność ALP i/lub GGT (zwykle 2–10 × ggn) oraz AST i ALT (zwykle 2–4 × ggn) w surowicy, zwiększone stężenie bilirubiny w surowicy (u ~60\% chorych prawidłowe w chwili rozpoznania), hipergammaglobulinemia (zwiększenie stężenia IgG i IgM u 45–80\% chorych)

    \item badania immunologiczne – autoprzeciwciała ANA i SMA (u 20–50\%), pANCA (u 30–80\%).
\end{enumerate}

    \item \textbf{MRCP\index{MRCP}:} złoty standard diagnostyczny; umożliwia wykazanie charakterystycznych zwężeń z następczym poszerzeniem dróg żółciowych (mogą wystąpić zarówno w odcinku wewnątrzwątrobowym, jak i zewnątrzwątrobowym).
    
    \item \textbf{ECPW:} badanie inwazyjne, powinno się je wykonywać wtedy, kiedy MRCP nie pozwala potwierdzić rozpoznania oraz w sytuacji klinicznej wymagającej bezpośredniej interwencji leczniczej (sfinkterotomia, protezowanie dróg żółciowych) i pobrania materiału do badania cytologicznego w razie podejrzenia raka dróg żółciowych.
    
    \item \textbf{Biopsja wątroby:} wskazana w razie podejrzenia PSC z zajęciem drobnych przewodzików, którego nie można potwierdzić w badaniach obrazowych, a także przy współistnieniu autoimmunologicznego zapalenia wątroby, gdy rozpoznanie na podstawie MRCP lub ECPW nie jest pewne oraz gdy istnieje kliniczne podejrzenie marskości wątroby (potwierdzenie zaawansowanego włóknienia ma istotne implikacje kliniczne).
\end{enumerate}

\subsection{Kryteria rozpoznania}

Typowy obraz dróg żółciowych w ECPW lub MRCP. 
Jeśli obraz w tych badaniach jest prawidłowy a obraz kliniczny wskazuje na PSC → badanie histologiczne wycinka wątroby (jeśli obraz histologiczny jest typowy dla PSC, rozpoznaje się wariant PSC z zajęciem tylko drobnych dróg żółciowych).

\subsection{Rozpoznanie różnicowe}

Stwardniające zapalenie dróg żółciowych związane z IgG4 (jedna z manifestacji choroby związanej z IgG4), inne przewlekłe choroby wątroby przebiegające z cholestazą (np. pierwotne zapalenie dróg żółciowych, zespół zanikających dróg żółciowych, eozynofilowe zapalenie dróg żółciowych), wtórne stwardniające zapalenie dróg żółciowych (np. po jatrogennym uszkodzeniu dróg żółciowych lub naczyń odpowiedzialnych za ich ukrwienie), wrodzone wady dróg żółciowych (np. zespół Alagille’a\index{Alagille’a zespół} [hipoplazja wewnątrzwątrobowych dróg żółciowych], zespół Carolego\index{Carolego zespół} [torbielowatość wewnątrzwątrobowych dróg żółciowych]), cholangiopatie infekcyjne (np. u chorych na AIDS, zarażenie chińską przywrą wątrobową).

\section{Leczenie}

\subsection{Leczenie farmakologiczne}

\begin{enumerate}
    \item Kwas ursodeoksycholowy (Proursan, Ursocam, Ursofalk, Ursopol, Ursoxyn) p.o. 13–15 mg/kg/d – prawdopodobnie zmniejsza ryzyko dysplazji i być może też raka jelita grubego u chorych na PSC ze współistniejącym WZJG.

    \item Leczenie świądu

    \item Profilaktyka i leczenie osteoporozy.

    \item Ostre zapalenie dróg żółciowych wymaga szybkiej i agresywnej antybiotykoterapii (chinolon lub cefalosporyna III generacji i metronidazol i.v.) z uwagi na ryzyko wystąpienia sepsy.
\end{enumerate}

\subsection{Leczenie inwazyjne}

\begin{enumerate}
    \item \textbf{Leczenie endoskopowe:} u chorych z dominującym zwężeniem dróg żółciowych można wykonać endoskopowe rozszerzanie balonem lub protezowanie dróg żółciowych (skuteczność 60–90\%).

    \item \textbf{Przeszczepienie wątroby:} wskazania – nawracające epizody ostrego zapalenia dróg żółciowych, brak poprawy po leczeniu farmakologicznym i endoskopowym zaawansowanych zwężeń dróg żółciowych, cechy opornej na leczenie zachowawcze schyłkowej niewydolności wątroby, oporny na leczenie świąd skóry. 
    Po transplantacji 5 lat przeżywa ~80\%. 
    PSC może nawrócić w przeszczepie.
\end{enumerate}

\section{Powikłania}

\begin{enumerate}
    \item zwiększone ryzyko rozwoju nowotworów złośliwych – rak dróg żółciowych (u 10–20\%), rak wątrobowokomórkowy (ryzyko zwiększone, gdy rozwinie się marskość wątroby), rak trzustki (ryzyko 14-krotnie większe niż w populacji ogólnej), rak jelita grubego (PSC dodatkowo zwiększa ryzyko związane z WZJG), rak pęcherzyka żółciowego (u 2\%)

    \item epizody ostrego zapalenia dróg żółciowych
\end{enumerate}

