
\chapter{Rak brodawki Vatera}

Raki brodawki większej dwunastnicy (Vatera) rozrastają się w miejscu połączenia dróg żółciowych z dwunastnicą. 
Czynnikiem ryzyka jest zespół rodzinnej polipowatości gruczolakowatej.

\section{Objawy}
Objawy jak w raku dróg żółciowych zlokalizowanym w obwodowej części zewnątrzwątrobowych dróg żółciowych, ale wcześniej pojawia się żółtaczka (stąd rozpoznanie w dość wczesnym stadium zaawansowania), może wystąpić niedrożność dwunastnicy.

\section{Rozpoznanie}
Duodenoskopia lub ECPW z pobraniem wycinków lub wymazu szczoteczkowego.

\section{Leczenie}
Pankreatoduodenektomia; we wczesnym stadium możliwa endoskopowa resekcja samej brodawki Vatera (ampulektomia). 
W przypadku zmiany nieresekcyjnej endoskopowa papillotomia i drenaż dróg żółciowych w celu złagodzeniach objawów cholestazy.