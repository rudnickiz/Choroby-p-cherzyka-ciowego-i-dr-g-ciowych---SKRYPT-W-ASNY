
\chapter{Rak dróg żółciowych}

\section{\DiEtp}

Najczęściej gruczolakorak (95\%) wywodzący się z nabłonka błony śluzowej dróg żółciowych. 

\textbf{Czynniki ryzyka:} pierwotne stwardniające zapalenie dróg żółciowych (PSC), wrzodziejące zapalenie jelita grubego, torbiele dróg żółciowych (w tym zespół Carolego), kamica przewodowa (zwłaszcza wtórna do przewlekłego zapalenia dróg żółciowych), infestacja przywrami Clonorchis sinensis, Opisthorchis viverrini, Opisthorchis felineus. 

\paragraph{Podział na podstawie lokalizacji:}

\begin{enumerate}
    \item rak wewnątrzwątrobowych dróg żółciowych

    \item rak zewnątrzwątrobowych dróg żółciowych – górny (guz Klatskina\index{Klatskina guz}; powyżej lub obejmujący miejsce podziału przewodu wątrobowego wspólnego na prawy i lewy przewód wątrobowy) i dolny.
\end{enumerate}

\section{\ObriP}

\textbf{Objawy:} żółtaczka cholestatyczna (jeśli rak zlokalizowany zewnątrzwątrobowo; w przypadku lokalizacji wewnątrzwątrobowej nie zawsze występuje), świąd skóry, dyskomfort i ból brzucha (zwykle w przypadku wewnątrzwątrobowej lokalizacji; stały, tępy, w okolicy podżebrowej prawej), utrata masy ciała, powiększenie wątroby, gorączka, wyczuwalny przez powłoki guz w prawym podżebrzu, powiększony, twardy, niebolesny pęcherzyk żółciowy (objaw Courvoisiera\index{Courvoisiera objaw} – w guzach zlokalizowanych poniżej połączenia przewodu pęcherzykowego z przewodem wątrobowym wspólnym).

Przebieg zwykle podstępny. 
Pojawienie się żółtaczki i świądu świadczy zwykle o znacznym zaawansowaniu raka; u większości chorych na tym etapie jest już nieresekcyjny, a czas przeżycia nie przekracza zwykle 12 mies. od chwili rozpoznania.

\section{Rozpoznanie}

\subsection{Badania pomocnicze}

\begin{enumerate}
    \item \textbf{Badania laboratoryjne:} w surowicy zwiększone stężenie bilirubiny (z przewagą sprzężonej), zwiększona aktywność ALP, GGT, ALT i AST; zwiększenie stężenia antygenów CA 19-9 (utrzymujące się po odbarczeniu żółtaczki mechanicznej) i CA 125.

    \item \textbf{Badania obrazowe:} 
    
    \begin{itemize}
        \item USG – badanie pierwszego wyboru w diagnostyce żółtaczki mechanicznej; może uwidocznić poszerzenie dróg żółciowych. 
    
    \item TK – umożliwia wykrycie zmiany ogniskowej i określenie zaawansowania nowotworu (wykrycie przerzutów do węzłów chłonnych, naciekania dużych naczyń i narządów sąsiednich). 
    
    \item MR – rozstrzygające badanie pozwalające na uwidocznienie guza. 
    
    \item MRCP – uwidocznienie zwężeń/poszerzeń dróg żółciowych dokładniej niż w ECPW, ocena rozległości guza. 
    
    \item EUS – dokładna ocena zewnątrzwątrobowych dróg żółciowych, pęcherzyka żółciowego, struktur we wnęce wątroby, regionalnych węzłów chłonnych i naczyń; biopsja guza lub węzłów chłonnych pod kontrolą EUS jest bardzo czułą metodą diagnostyczną.
    
    \item ECPW – umożliwia pobranie wycinka lub wymazu szczoteczkowego i/lub wprowadzenie protezy do dróg żółciowych poprzez miejsce zwężenia w celu poprawy odpływu żółci. 
    
    \item Cholangioskopia – umożliwia wizualizację dróg żółciowych, pobranie wymazu szczoteczkowego lub wycinka ze zmiany. 
    
    \item PET – rozpoznawanie przerzutów odległych, monitorowanie leczenia i wznowy.
    \end{itemize}

    \item \textbf{Badanie mikroskopowe:} materiał pobrany podczas ECPW, cholangioskopii lub biopsji igłowej pod kontrolą USG, TK lub EUS.
\end{enumerate}

\subsection{Kryteria rozpoznania}

Rak głowy trzustki, brodawki Vatera, dwunastnicy lub pęcherzyka żółciowego; zwężenie dróg żółciowych (zwykle pooperacyjne); pierwotne stwardniające zapalenie dróg żółciowych; kamica dróg żółciowych; zespół Mirizziego; przerzuty do wątroby; stwardniające zapalenie dróg żółciowych związane z IgG4.

\section{Leczenie}

\begin{enumerate}
    \item \textbf{Leczenie operacyjne:} rak wewnątrzwątrobowy → hemihepatektomia; rak zewnątrzwątrobowych dróg żółciowych → zakres resekcji zależy od położenia guza względem wnęki wątroby i głowy trzustki (resekcja odcinkowa dróg żółciowych, ew. poszerzona o odpowiednie segmenty wątroby, hemihepatektomia, a w przypadku dystalnego położenia zmiany – pankreatoduodenektomia). Po operacji radykalnej w przypadku korzystnego (obwodowego) położenia guza 5 lat przeżywa do 40% chorych.

    \item \textbf{Adiuwantowa radioterapia i chemioterapia:} Adiuwantowa radioterapia i chemioterapia nie wydłużają przeżycia; w niektórych ośrodkach stosuje się radioterapię i chemioterapię (np. gemcytabinę z cisplatyną lub oksaliplatyną).

    \item \textbf{Leczenie paliatywne:} w celu zmniejszenia objawów cholestazy drenaż dróg żółciowych metodą endoskopową, z wszczepieniem stentu, rzadziej drenaż przezskórny lub operacyjny. 
    U chorych z przeciwwskazaniami do resekcji ablacja guza (np. mikrofalowa) może zahamować progresję choroby. 
    Rokowanie tym gorsze, im bardziej dowątrobowo rozwija się rak dróg zewnątrzwątrobowych.
\end{enumerate}