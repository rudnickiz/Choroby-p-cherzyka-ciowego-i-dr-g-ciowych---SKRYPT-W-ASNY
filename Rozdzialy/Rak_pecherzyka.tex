
\chapter{Rak pęcherzyka żółciowego}

\section{\DiEtp}

Najczęściej jest to gruczolakorak (\textit{adenocarcinoma}) wywodzący się z nabłonka błony śluzowej pęcherzyka żółciowego.

\textbf{Czynniki ryzyka:} wieloletnia kamica pęcherzyka żółciowego (zwłaszcza złogi $>$3 cm; kamienie żółciowe w pęcherzyku w ~80\% przypadków raka; raka rozpoznaje się w ~1\% pęcherzyków usuwanych z powodu kamicy), torbiele dróg żółciowych, polip pęcherzyka żółciowego $>$1 cm, zespoły rodzinnej polipowatości. 

\section{\ObriP}

Objawy niecharakterystyczne, występują zwykle późno (często rozpoznanie już w stadium nieoperacyjnym), najczęściej ból w okolicy podżebrowej prawej, tępy, promieniujący w prawą stronę do kręgosłupa i okolicy międzyłopatkowej; żółtaczka i świąd (w wyniku naciekania przez nowotwór przewodu żółciowego; pogarsza rokowanie); nudności i wymioty; utrata łaknienia, utrata masy ciała; guz wyczuwalny w prawym górnym kwadrancie brzucha.

\section{Rozpoznanie}

\subsection{Badania pomocnicze}

\begin{enumerate}
    \item \textbf{Badania laboratoryjne:} w surowicy zwiększona aktywność ALP i GGT, stężenie bilirubiny, aktywność ALT i AST (w fazie znacznego zaawansowania choroby, naciekania wątroby i zmian przerzutowych); zwiększenie stężenia CEA i CA19-9.

    \item \textbf{Badania obrazowe:} USG – może wykazać pogrubienie ściany ($>$1 cm), guz w świetle lub na zewnątrz pęcherzyka (polipy pęcherzyka o średnicy $<$1 cm rzadko są rakiem). 
    
    EUS – określenie głębokości naciekania raka oraz ocena zajęcia regionalnych węzłów chłonnych. 
    
    TK – zmiany podobne jak w USG; dodatkowo ocena topografii zmian, otaczających węzłów chłonnych i naciekania wątroby. 
    
    MRCP i ECPW – mogą ujawnić zwężenia środkowego odcinka dróg żółciowych, co wskazuje na ich naciekanie przez raka.
\end{enumerate}

\subsection{Kryteria rozpoznania}

Podejrzenie na podstawie wyników badań obrazowych; pewne rozpoznanie na podstawie badania histologicznego usuniętego pęcherzyka żółciowego.

\section{Leczenie}

\begin{enumerate}
    \item \textbf{Leczenie z intencją wyleczenia:} leczenie operacyjne (zabieg radykalny możliwy tylko we wczesnej fazie choroby) – cholecystektomia metodą otwartą (wycięcie pęcherzyka z szerokim [$\geq$2 cm] marginesem wątroby + usunięcie regionalnych węzłów chłonnych). 
    
    Jeśli niezaawansowanego miejscowo raka pęcherzyka rozpoznano przypadkowo w badaniu histologicznym przeprowadzonym po cholecystektomii laparoskopowej, nie ma uzasadnienia dla otwartej reoperacji.

    \item \textbf{Leczenie paliatywne:} drenaż dróg żółciowych metodą endoskopową, z wszczepieniem stentu, w celu zmniejszenia objawów cholestazy.
\end{enumerate}

\section{Rokowanie}

Średni czas przeżycia ~6 mies. 
Czas przeżycia po operacji zależy od stopnia zaawansowania klinicznego: od $>$90\% w stopniu I do maks. 15\% 5-letnich w stopniu IV (wg TNM).