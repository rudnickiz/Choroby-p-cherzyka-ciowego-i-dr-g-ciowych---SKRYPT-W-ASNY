
\chapter{Ostre zapalenie pęcherzyka żółciowego}

\section{\DiEtp}

Najczęstsze powikłanie kamicy pęcherzyka żółciowego w następstwie upośledzenia odpływu żółci z pęcherzyka żółciowego albo zamknięcia światła lub obrzęku błony śluzowej przewodu pęcherzykowego. 
W ~10\% przypadków ma charakter niekamiczy, zwykle w ciężkich chorobach ogólnoustrojowych.

\section{Obraz kliniczny}

\paragraph{Objawy:}
kolka żółciowa utrzymująca się $>$5 h, gorączka i dreszcze, wymioty, silna tkliwość powłok brzusznych w okolicy podżebrowej prawej, dodatni objaw Murphy’ego\index{Murphy’ego objaw} (podczas głębokiego wdechu wykonywanego przez pacjenta z ręką badającego przyłożoną pod prawym łukiem żebrowym występuje ból powodujący przerwanie wdechu) i objaw Chełmońskiego, niekiedy wyczuwalny bolesny pęcherzyk żółciowy, objawy otrzewnowe (u niektórych chorych), przyśpieszenie tętna i oddechu.

\section{Rozpoznanie}

\subsection{Badania pomocnicze}

\begin{enumerate}
    \item\textbf{ Badania laboratoryjne:} leukocytoza z przesunięciem obrazu odsetkowego leukocytów w lewo, zwiększone stężenie CRP w surowicy, niekiedy zwiększona aktywność AST, ALT, ALP i amylazy w surowicy, hiperbilirubinemia.

    \item \textbf{Badania obrazowe:} USG – objawy duże: obecność złogów, obrzęk ściany pęcherzyka, obecność gazu w ścianie (zapalenie zgorzelinowe), dodatni ultrasonograficzny (wywołany uciskiem głowicą) objaw Murphy’ego; objawy małe: powiększenie pęcherzyka, pogrubienie ściany, zmiany w świetle pęcherzyka (np. szlam), okołopęcherzykowe zbiorniki płynowe. 
    TK – jest pomocna w ustaleniu rozpoznania u chorych z niekamiczym ostrym zapaleniem pęcherzyka żółciowego, a także w wykrywaniu powikłań.
\end{enumerate}

\subsection{Kryteria rozpoznania}

Objawy podmiotowe i przedmiotowe oraz obraz USG.


\section{Leczenie}

\begin{enumerate}
    \item Ścisła dieta.

    \item \textbf{Nawadnianie chorego:} wlew i.v. 0,9\% NaCl.

    \item \textbf{Leki przeciwbólowe i rozkurczowe:} jak w kolce żółciowej.

    \item Leczenie empiryczne antybiotykiem o szerokim spektrum działania, np. cefalosporyną III generacji (ceftriakson, cefoperazon), cyprofloksacyną, w razie podejrzenia zakażenia bakteriami beztlenowymi dodaj metronidazol lub amoksycylinę z kwasem klawulanowym. 
    Leczenie w przypadku niepowikłanego przebiegu pooperacyjnego może trwać 5–7 dni.

    \item \textbf{Cholecystektomia (preferowana laparoskopowa):} w każdym przypadku ostrego kamiczego zapalenia pęcherzyka żółciowego w ciągu 72 h od przyjęcia do szpitala (jeśli chory nie może być operowany w ciągu 1 tyg. od wystąpienia objawów, operację należy odroczyć o $\geq$6 tyg.).
\end{enumerate}

\section{Powikłania}

Ropniak, martwica lub perforacja (ograniczona lub z rozlanym żółciowym zapaleniem otrzewnej) pęcherzyka żółciowego (wymagają pilnej interwencji chirurgicznej), wodniak pęcherzyka żółciowego, ropień wątroby, przetoka pomiędzy pęcherzykiem żółciowym a jelitem (przedostanie się dużych złogów do jelita może spowodować żółciową niedrożność jelit), zespół Mirizziego\index{Miriziego zespół} (zablokowanie dużego złogu w szyjce pęcherzyka lub w przewodzie pęcherzykowym w miejscu jego połączenia z przewodem wątrobowym wspólnym powoduje objawy ucisku na przewód żółciowy wspólny).