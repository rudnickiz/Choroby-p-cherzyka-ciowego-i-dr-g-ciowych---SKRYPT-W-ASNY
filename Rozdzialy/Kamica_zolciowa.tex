
\chapter{Kamica żółciowa}

\section{\DiEtp}

Obecność złogów w żółci. 

\paragraph{Podział ze względu na umiejscowienie złogów:}

\begin{enumerate}
    \item kamica pęcherzyka żółciowego – złogi tworzą się w pęcherzyku żółciowym

    \item kamica przewodowa – złogi znajdują się w drogach żółciowych zewnątrzwątrobowych lub wewnątrzwątrobowych i mogą pochodzić z pęcherzyka żółciowego lub powstawać w przewodach żółciowych (pierwotna kamica przewodowa; rzadka w Europie i Ameryce Północnej); w 95\% przypadków współistnieje kamica pęcherzyka żółciowego.
\end{enumerate}

Złogi w zależności od składu dzieli się na: cholesterolowe (żółte lub żółtobrunatne), barwnikowe (rzadkie w Europie i Ameryce Północnej) i mieszane.

\paragraph{Czynniki ryzyka kamicy cholesterolowej:}
Czynniki genetyczne, płeć żeńska (4 razy częściej niż u mężczyzn), ciąża, wiek $>$40 lat, cukrzyca, otyłość, mukowiscydoza, hipertriglicerydemia, leki (estrogeny, doustne środki antykoncepcyjne, fibraty, ceftriakson, somatostatyna i jej analogi), szybka utrata masy ciała (np. po operacji bariatrycznej lub diecie bardzo niskokalorycznej). 

\paragraph{Czynniki ryzyka kamicy barwnikowej:}
Niedokrwistość hemolityczna, choroba Leśniowskiego i Crohna, marskość wątroby, długotrwałe całkowite żywienie pozajelitowe.