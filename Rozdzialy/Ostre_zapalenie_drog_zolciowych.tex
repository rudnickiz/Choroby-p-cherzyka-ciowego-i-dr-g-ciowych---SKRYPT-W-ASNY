
\chapter{Ostre zapalenie dróg żółciowych}

\section{\DiEtp}

Odcinkowy lub rozlany ostry proces zapalny wewnątrz- i/lub zewnątrzwątrobowych dróg żółciowych spowodowany zakażeniem w następstwie utrudnienia lub zablokowania odpływu żółci. 

\textbf{Czynniki etiologiczne:} najczęściej Escherichia coli, Klebsiella, Enterococcus, Enterobacter, Streptococcus i Pseudomonas aeruginosa, 15\% bakterie beztlenowe. 

Czynniki ryzyka – przyczyny cholestazy: kamica żółciowa, nowotwory utrudniające odpływ żółci (naciek przewodów lub brodawki większej, ucisk z zewnątrz), zwężenie dróg żółciowych pozapalne i jatrogenne, pierwotne stwardniające zapalenie dróg żółciowych, ucisk na przewody żółciowe przez torbiel trzustki lub powiększone węzły chłonne.

\section{\ObriP}

Typowe objawy kliniczne to triada Charcota\index{Charcota triada}: silny ból o charakterze kolki żółciowej w okolicy podżebrowej prawej lub w nadbrzuszu środkowym, gorączka z dreszczami i żółtaczka. 
Gdy dodatkowo występują objawy wstrząsu i zaburzenia świadomości – pentada Reynoldsa\index{Reynoldsa pentada}. 
Ponadto w badaniu przedmiotowym bolesność palpacyjna w okolicy podżebrowej prawej, ze wzmożonym napięciem mięśni brzusznych. 
Może się rozwinąć wstrząs septyczny. 
Nieleczone ostre bakteryjne zapalenie dróg żółciowych zazwyczaj kończy się zgonem.

\section{Rozpoznanie}

\subsection{Badania pomocnicze}

\begin{enumerate}
    \item \textbf{Badania laboratoryjne:} jak w kamicy przewodowej + leukocytoza z przesunięciem obrazu odsetkowego w lewo oraz duże stężenie CRP w surowicy. 
    W ciężkim zapaleniu mogą wystąpić cechy sepsy.

    \item \textbf{Badania obrazowe:} USG – może uwidocznić poszerzenie dróg żółciowych wewnątrz- i zewnątrzwątrobowych, kamicę przewodową. 
    
    ECPW w trybie pilnym – jest to najlepsza metoda diagnostyczna w ostrym zapaleniu dróg żółciowych, umożliwiająca jednocześnie interwencję leczniczą.
\end{enumerate}

\subsection{Kryteria rozpoznania}

Najczęściej stosuje się kryteria z Tokio, w których bierze się pod uwagę:
\begin{itemize}
    \item A – ogólnoustrojową reakcję zapalną (A1 – gorączka [$>$38$\degree$C] i/lub silne dreszcze, A2 – laboratoryjne wskaźniki zapalenia: zwiększona lub zmniejszona liczba leukocytów [$<$4000/$\mu$l albo $>$10 000/$\mu$l], stężenie CRP $>$1 mg/dl lub inne wskaźniki stanu zapalnego)

    \item B – cholestazę (B1 – stężenie bilirubiny $\geq$2 mg/dl [34 $\mu$mol/l], B2 – aktywność ALP, GGT, AST i ALT $>$1,5 × ggn)

    \item C – badania obrazowe (C1 – poszerzenie dróg żółciowych, C2 – przyczyna widoczna w badaniach obrazowych [zwężenie, kamień, stent]).
\end{itemize}

Rozpoznanie jest pewne, gdy spełnione jest po 1 kryterium A, B i C; a rozpoznanie jest prawdopodobne, gdy spełnione jest 1 kryterium A oraz 1 kryterium B lub C.

\paragraph{Określenie ciężkości ostrego zapalenia dróg żółciowych według wytycznych z Tokio:\index{Tokio wytyczne}}

\begin{enumerate}
    \item ciężkie (stopień III) – występuje zaburzenie czynności $\geq$1 układu: sercowo-naczyniowego (hipotensja wymagająca stosowania dopaminy i.v. w dawce $\geq$5 $\mu$g/kg/min lub noradrenaliny niezależnie od dawki), oddechowego (PaO2/FiO2 $<$300), nerwowego (zaburzenia przytomności), krwiotwórczego (liczba płytek krwi $<$100 000/$\mu$l), lub narządu: nerek (skąpomocz, stężenie kreatyniny w surowicy $>$2 mg/dl [177 $\mu$mol/l]), wątroby (INR $>$1,5)

    \item umiarkowane (stopień II) – występują $\geq$2 z wymienionych: liczba leukocytów we krwi obwodowej $>$12 000/$\mu$l albo $<$4000/$\mu$l; gorączka ($\geq$39$\degree$C); wiek $\geq$75 lat; stężenie bilirubiny w surowicy $\geq$5 mg/dl (85 $\mu$mol/l), stężenie albuminy w osoczu $<$0,7 × dolna granica normy

    \item łagodne (stopień I) – nie są spełnione kryteria stopnia III ani II.
\end{enumerate}

\subsection{Rozpoznanie różnicowe}

Inne przyczyny gorączki i bólu w nadbrzuszu: ostre zapalenie pęcherzyka żółciowego, ostre żółciopochodne zapalenie trzustki, ropień wątroby, ostre wirusowe zapalenie wątroby, zapalenie uchyłków, perforacja jelita.

\section{Leczenie}

Każdego chorego należy ocenić pod kątem występowania ciężkiej sepsy i wdrożyć odpowiednie postępowanie.

\subsection{Leczenie zachowawcze}

\begin{enumerate}
    \item Ścisła dieta.

\item \textbf{Nawadnianie chorego:} wlew i.v. 0,9\% NaCl.

\item \textbf{Leki przeciwbólowe i rozkurczowe:} jak w kolce żółciowej.

\item \textbf{Leczenie empiryczne antybiotykiem o szerokim spektrum działania:} aktywne wobec bakterii Gram-ujemnych i beztlenowych, np. cyprofloksacyna z metronidazolem; stężenie terapeutyczne w surowicy i w żółci osiągają także np. cefalosporyny II i III generacji, imipenem, aminoglikozydy, ale tylko cyprofloksacyna zachowuje stężenie lecznicze w żółci u chorych z niedrożnością dróg żółciowych.
\end{enumerate}

\subsection{Leczenie inwazyjne}

\begin{enumerate}
    \item ECPW ze sfinkterotomią endoskopową z usunięciem złogów i/lub  protezowaniem dróg żółciowych umożliwiającym prawidłowy drenaż żółci – metoda z wyboru. 
    Czas wykonania ECPW zależy od stopnia ciężkości ostrego zapalenia dróg żółciowych (ocena → patrz wyżej): ciężkie – możliwie najszybciej, w ciągu 12 h u chorego ze wstrząsem septycznym; umiarkowane – w ciągu 48–72 h; łagodne – zabieg planowy.

    \item Drenaż przezskórny pod kontrolą USG lub TK – rozważyć, gdy nie można wykonać ECPW lub nie udało się udrożnić dróg żółciowych.
\end{enumerate}