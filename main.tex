\documentclass[polish,a4paper,10pt]{book}
\usepackage[top=2cm,bottom=3cm,left=2cm,right=2cm]{geometry} 

\usepackage{minitoc} 
\usepackage{fancyhdr}

%\RedeclareSectionCommand[%
%tocindent=9em,tocnumwidth=7em,]{section}

\usepackage{lmodern} 
\usepackage[utf8]{inputenc}
\usepackage[OT4]{polski}

%kolorowanie naglowkow tabel
\usepackage{color}
\usepackage{colortbl}


% jak ominąć datę po stronie tytułowej to wpisać jak niżej
\date{\vspace{-5ex}}

\usepackage{indentfirst}

% jak ominąć datę po stronie tytułowej to wpisać jak niżej
\date{\vspace{-5ex}}
\usepackage{indentfirst}
\setlength{\parindent}{1cm}
\usepackage{graphicx}   % need for figures
\usepackage{color, colortbl}   % use if color is used in text and tables !!!
\usepackage{subfigure}  % use for side-by-side figures
\usepackage{gensymb }
\usepackage{amssymb}
\usepackage{makeidx}
\usepackage{booktabs}

\usepackage{longtable}
\usepackage{multirow}
\usepackage{threeparttable}

% zeby działały linki do sidewaysfigures!!
\usepackage[hypcap=true]{caption}  % <==============
\usepackage{rotating}
\usepackage[pdftex]{hyperref}
% zeby działały linki do sidewaysfigures!!

\newcommand*{\DiEtp}{Definicja i etiopatogeneza}
\newcommand*{\ObriP}{Obraz kliniczny i przebieg naturalny}
\newcommand*{\Tx}{Leczenie}


\title{Choroby pęcherzyka żółciowego i dróg żółciowych - skrypt własny}
\author{Zbigniew Rudnicki}
\makeindex
\begin{document}
	\maketitle
	
	%\listoftables
	%\listoffigures
	\tableofcontents
	

\chapter{Zaburzenia czynnościowe pęcherzyka żółciowego i dysfunkcja zwieracza Oddiego}

\section{Definicja i obraz kliniczny}

Choroby czynnościowe pęcherzyka żółciowego i zwieracza Oddiego można rozpoznać w przypadku występowania bólu żółciowego lub nawracającego ostrego zapalenia trzustki (OZT) u chorych bez uchwytnej choroby organicznej mogącej być ich przyczyną.

Ból żółciowy to wg wytycznych rzymskich IV ból zlokalizowany w nadbrzuszu i/lub w prawym górnym kwadrancie brzucha, który spełnia wszystkie kryteria:

\begin{enumerate}
    \item osiąga stałe natężenie i trwa $\geq$30 minut

    \item pojawia się w różnych odstępach czasu (nie codziennie)

    \item jest na tyle silny, że zaburza normalną aktywność lub prowadzi do wizyt na oddziale ratunkowym

    \item nie jest istotnie ($<$20\% epizodów bólu) związany z defekacją

    \item nie jest istotnie łagodzony przez zmiany pozycji ciała ani przez leki zobojętniające lub hamujące wydzielanie kwasu solnego.
\end{enumerate}

\textbf{Bólowi żółciowemu mogą towarzyszyć:} nudności i wymioty, promieniowanie do pleców i/lub do prawej okolicy podłopatkowej, budzenie się chorego ze snu.

Ból żółciowy u chorych z pęcherzykiem żółciowym, u których nie stwierdza się choroby organicznej, może wskazywać na zaburzenie czynnościowe pęcherzyka żółciowego, natomiast u chorych po cholecystektomii można rozpoznać czynnościową dysfunkcję żółciowego zwieracza Oddiego.

Nawracające idiopatyczne OZT może wskazywać na czynnościową dysfunkcję trzustkowego zwieracza Oddiego.

\section{Rozpoznanie}

\subsection{Kryteria rozpoznania}

\begin{enumerate}
    \item Zaburzenie czynnościowe pęcherzyka żółciowego

\paragraph{Kryteria wymagane:}

\begin{enumerate}
    \item ból żółciowy

    \item niewystępowanie kamieni żółciowych (też tzw. błotka żółciowego) ani innych nieprawidłowości organicznych.
\end{enumerate}

\paragraph{Kryteria dodatkowe (nie muszą być spełnione):}

\begin{enumerate}
    \item zmniejszona frakcja wyrzutowa w scyntygrafii pęcherzyka żółciowego

    \item prawidłowa aktywność / stężenie we krwi enzymów wątrobowych, bilirubiny związanej i amylazy/lipazy (jeśli aktywność enzymów wątrobowych jest zwiększona z innej znanej przyczyny, np. niealkoholowej stłuszczeniowej choroby wątroby, to nie wyklucza zaburzenia czynnościowego pęcherzyka żółciowego).
\end{enumerate}

\item Czynnościowa dysfunkcja żółciowego zwieracza Oddiego

\paragraph{Kryteria wymagane:}

\begin{enumerate}
    \item ból żółciowy

    \item wzrost aktywności enzymów wątrobowych albo poszerzenie dróg żółciowych (nie oba objawy jednocześnie)

    \item niewystępowanie kamicy przewodowej i innych nieprawidłowości organicznych.
\end{enumerate}

\paragraph{Kryteria dodatkowe (nie muszą być spełnione):}

\begin{enumerate}
    \item prawidłowa aktywność amylazy/lipazy (może jednak być zwiększona przy niektórych napadach bólu)

    \item nieprawidłowy wynik manometrii zwieracza Oddiego (wykonuje się tylko u niektórych chorych)

    \item scyntygrafia wątroby i dróg żółciowych – przydatność wątpliwa; nie powinna wskazywać na niedrożność dróg żółciowych, często wykazuje zwolnione wydzielanie radionuklidu do żółci i/lub zwolniony przepływ żółci z wnęki wątroby do dwunastnicy.
\end{enumerate}

\item Czynnościowa dysfunkcja trzustkowego zwieracza Oddiego

\paragraph{Kryteria wymagane:}

\begin{enumerate}
    \item udokumentowane nawracające epizody OZT (typowy ból z $>$3-krotnym wzrostem aktywności amylazy lub lipazy w stosunku do normy i/lub cechy OZT w badaniach obrazowych)

    \item wykluczenie OZT o innej etiologii

    \item prawidłowy obraz w EUS

    \item nieprawidłowy wynik manometrii zwieracza.
\end{enumerate}
\end{enumerate}

\subsection{Rozpoznanie różnicowe}

Inne przyczyny bólu brzucha, przede wszystkim kamica żółciowa, choroba wrzodowa, przewlekłe zapalenie trzustki, dyspepsja czynnościowa, zespół jelita drażliwego, powikłania pooperacyjne.

Wstępna diagnostyka powinna obejmować badania biochemiczne wątroby i trzustki, gastroskopię i badania obrazowe jamy brzusznej – najlepiej EUS i MRCP. 

EUS\index{EUS} jest najdokładniejszym badaniem nieinwazyjnym pozwalającym wykluczyć kamicę przewodową i patologię brodawki Vatera. 
Nie wykonuj ECPW, jeśli nie ma wyraźnych obiektywnych cech zastoju żółci.


\chapter{Kamica żółciowa}

\section{\DiEtp}

Obecność złogów w żółci. 

\paragraph{Podział ze względu na umiejscowienie złogów:}

\begin{enumerate}
    \item kamica pęcherzyka żółciowego – złogi tworzą się w pęcherzyku żółciowym

    \item kamica przewodowa – złogi znajdują się w drogach żółciowych zewnątrzwątrobowych lub wewnątrzwątrobowych i mogą pochodzić z pęcherzyka żółciowego lub powstawać w przewodach żółciowych (pierwotna kamica przewodowa; rzadka w Europie i Ameryce Północnej); w 95\% przypadków współistnieje kamica pęcherzyka żółciowego.
\end{enumerate}

Złogi w zależności od składu dzieli się na: cholesterolowe (żółte lub żółtobrunatne), barwnikowe (rzadkie w Europie i Ameryce Północnej) i mieszane.

\paragraph{Czynniki ryzyka kamicy cholesterolowej:}
Czynniki genetyczne, płeć żeńska (4 razy częściej niż u mężczyzn), ciąża, wiek $>$40 lat, cukrzyca, otyłość, mukowiscydoza, hipertriglicerydemia, leki (estrogeny, doustne środki antykoncepcyjne, fibraty, ceftriakson, somatostatyna i jej analogi), szybka utrata masy ciała (np. po operacji bariatrycznej lub diecie bardzo niskokalorycznej). 

\paragraph{Czynniki ryzyka kamicy barwnikowej:}
Niedokrwistość hemolityczna, choroba Leśniowskiego i Crohna, marskość wątroby, długotrwałe całkowite żywienie pozajelitowe.

\chapter{Ostre zapalenie pęcherzyka żółciowego}

\section{\DiEtp}

Najczęstsze powikłanie kamicy pęcherzyka żółciowego w następstwie upośledzenia odpływu żółci z pęcherzyka żółciowego albo zamknięcia światła lub obrzęku błony śluzowej przewodu pęcherzykowego. 
W ~10\% przypadków ma charakter niekamiczy, zwykle w ciężkich chorobach ogólnoustrojowych.

\section{Obraz kliniczny}

\paragraph{Objawy:}
kolka żółciowa utrzymująca się $>$5 h, gorączka i dreszcze, wymioty, silna tkliwość powłok brzusznych w okolicy podżebrowej prawej, dodatni objaw Murphy’ego\index{Murphy’ego objaw} (podczas głębokiego wdechu wykonywanego przez pacjenta z ręką badającego przyłożoną pod prawym łukiem żebrowym występuje ból powodujący przerwanie wdechu) i objaw Chełmońskiego, niekiedy wyczuwalny bolesny pęcherzyk żółciowy, objawy otrzewnowe (u niektórych chorych), przyśpieszenie tętna i oddechu.

\section{Rozpoznanie}

\subsection{Badania pomocnicze}

\begin{enumerate}
    \item\textbf{ Badania laboratoryjne:} leukocytoza z przesunięciem obrazu odsetkowego leukocytów w lewo, zwiększone stężenie CRP w surowicy, niekiedy zwiększona aktywność AST, ALT, ALP i amylazy w surowicy, hiperbilirubinemia.

    \item \textbf{Badania obrazowe:} USG – objawy duże: obecność złogów, obrzęk ściany pęcherzyka, obecność gazu w ścianie (zapalenie zgorzelinowe), dodatni ultrasonograficzny (wywołany uciskiem głowicą) objaw Murphy’ego; objawy małe: powiększenie pęcherzyka, pogrubienie ściany, zmiany w świetle pęcherzyka (np. szlam), okołopęcherzykowe zbiorniki płynowe. 
    TK – jest pomocna w ustaleniu rozpoznania u chorych z niekamiczym ostrym zapaleniem pęcherzyka żółciowego, a także w wykrywaniu powikłań.
\end{enumerate}

\subsection{Kryteria rozpoznania}

Objawy podmiotowe i przedmiotowe oraz obraz USG.


\section{Leczenie}

\begin{enumerate}
    \item Ścisła dieta.

    \item \textbf{Nawadnianie chorego:} wlew i.v. 0,9\% NaCl.

    \item \textbf{Leki przeciwbólowe i rozkurczowe:} jak w kolce żółciowej.

    \item Leczenie empiryczne antybiotykiem o szerokim spektrum działania, np. cefalosporyną III generacji (ceftriakson, cefoperazon), cyprofloksacyną, w razie podejrzenia zakażenia bakteriami beztlenowymi dodaj metronidazol lub amoksycylinę z kwasem klawulanowym. 
    Leczenie w przypadku niepowikłanego przebiegu pooperacyjnego może trwać 5–7 dni.

    \item \textbf{Cholecystektomia (preferowana laparoskopowa):} w każdym przypadku ostrego kamiczego zapalenia pęcherzyka żółciowego w ciągu 72 h od przyjęcia do szpitala (jeśli chory nie może być operowany w ciągu 1 tyg. od wystąpienia objawów, operację należy odroczyć o $\geq$6 tyg.).
\end{enumerate}

\section{Powikłania}

Ropniak, martwica lub perforacja (ograniczona lub z rozlanym żółciowym zapaleniem otrzewnej) pęcherzyka żółciowego (wymagają pilnej interwencji chirurgicznej), wodniak pęcherzyka żółciowego, ropień wątroby, przetoka pomiędzy pęcherzykiem żółciowym a jelitem (przedostanie się dużych złogów do jelita może spowodować żółciową niedrożność jelit), zespół Mirizziego\index{Miriziego zespół} (zablokowanie dużego złogu w szyjce pęcherzyka lub w przewodzie pęcherzykowym w miejscu jego połączenia z przewodem wątrobowym wspólnym powoduje objawy ucisku na przewód żółciowy wspólny).

\chapter{Ostre zapalenie dróg żółciowych}

\section{\DiEtp}

Odcinkowy lub rozlany ostry proces zapalny wewnątrz- i/lub zewnątrzwątrobowych dróg żółciowych spowodowany zakażeniem w następstwie utrudnienia lub zablokowania odpływu żółci. 

\textbf{Czynniki etiologiczne:} najczęściej Escherichia coli, Klebsiella, Enterococcus, Enterobacter, Streptococcus i Pseudomonas aeruginosa, 15\% bakterie beztlenowe. 

Czynniki ryzyka – przyczyny cholestazy: kamica żółciowa, nowotwory utrudniające odpływ żółci (naciek przewodów lub brodawki większej, ucisk z zewnątrz), zwężenie dróg żółciowych pozapalne i jatrogenne, pierwotne stwardniające zapalenie dróg żółciowych, ucisk na przewody żółciowe przez torbiel trzustki lub powiększone węzły chłonne.

\section{\ObriP}

Typowe objawy kliniczne to triada Charcota\index{Charcota triada}: silny ból o charakterze kolki żółciowej w okolicy podżebrowej prawej lub w nadbrzuszu środkowym, gorączka z dreszczami i żółtaczka. 
Gdy dodatkowo występują objawy wstrząsu i zaburzenia świadomości – pentada Reynoldsa\index{Reynoldsa pentada}. 
Ponadto w badaniu przedmiotowym bolesność palpacyjna w okolicy podżebrowej prawej, ze wzmożonym napięciem mięśni brzusznych. 
Może się rozwinąć wstrząs septyczny. 
Nieleczone ostre bakteryjne zapalenie dróg żółciowych zazwyczaj kończy się zgonem.

\section{Rozpoznanie}

\subsection{Badania pomocnicze}

\begin{enumerate}
    \item \textbf{Badania laboratoryjne:} jak w kamicy przewodowej + leukocytoza z przesunięciem obrazu odsetkowego w lewo oraz duże stężenie CRP w surowicy. 
    W ciężkim zapaleniu mogą wystąpić cechy sepsy.

    \item \textbf{Badania obrazowe:} USG – może uwidocznić poszerzenie dróg żółciowych wewnątrz- i zewnątrzwątrobowych, kamicę przewodową. 
    
    ECPW w trybie pilnym – jest to najlepsza metoda diagnostyczna w ostrym zapaleniu dróg żółciowych, umożliwiająca jednocześnie interwencję leczniczą.
\end{enumerate}

\subsection{Kryteria rozpoznania}

Najczęściej stosuje się kryteria z Tokio, w których bierze się pod uwagę:
\begin{itemize}
    \item A – ogólnoustrojową reakcję zapalną (A1 – gorączka [$>$38$\degree$C] i/lub silne dreszcze, A2 – laboratoryjne wskaźniki zapalenia: zwiększona lub zmniejszona liczba leukocytów [$<$4000/$\mu$l albo $>$10 000/$\mu$l], stężenie CRP $>$1 mg/dl lub inne wskaźniki stanu zapalnego)

    \item B – cholestazę (B1 – stężenie bilirubiny $\geq$2 mg/dl [34 $\mu$mol/l], B2 – aktywność ALP, GGT, AST i ALT $>$1,5 × ggn)

    \item C – badania obrazowe (C1 – poszerzenie dróg żółciowych, C2 – przyczyna widoczna w badaniach obrazowych [zwężenie, kamień, stent]).
\end{itemize}

Rozpoznanie jest pewne, gdy spełnione jest po 1 kryterium A, B i C; a rozpoznanie jest prawdopodobne, gdy spełnione jest 1 kryterium A oraz 1 kryterium B lub C.

\paragraph{Określenie ciężkości ostrego zapalenia dróg żółciowych według wytycznych z Tokio:\index{Tokio wytyczne}}

\begin{enumerate}
    \item ciężkie (stopień III) – występuje zaburzenie czynności $\geq$1 układu: sercowo-naczyniowego (hipotensja wymagająca stosowania dopaminy i.v. w dawce $\geq$5 $\mu$g/kg/min lub noradrenaliny niezależnie od dawki), oddechowego (PaO2/FiO2 $<$300), nerwowego (zaburzenia przytomności), krwiotwórczego (liczba płytek krwi $<$100 000/$\mu$l), lub narządu: nerek (skąpomocz, stężenie kreatyniny w surowicy $>$2 mg/dl [177 $\mu$mol/l]), wątroby (INR $>$1,5)

    \item umiarkowane (stopień II) – występują $\geq$2 z wymienionych: liczba leukocytów we krwi obwodowej $>$12 000/$\mu$l albo $<$4000/$\mu$l; gorączka ($\geq$39$\degree$C); wiek $\geq$75 lat; stężenie bilirubiny w surowicy $\geq$5 mg/dl (85 $\mu$mol/l), stężenie albuminy w osoczu $<$0,7 × dolna granica normy

    \item łagodne (stopień I) – nie są spełnione kryteria stopnia III ani II.
\end{enumerate}

\subsection{Rozpoznanie różnicowe}

Inne przyczyny gorączki i bólu w nadbrzuszu: ostre zapalenie pęcherzyka żółciowego, ostre żółciopochodne zapalenie trzustki, ropień wątroby, ostre wirusowe zapalenie wątroby, zapalenie uchyłków, perforacja jelita.

\section{Leczenie}

Każdego chorego należy ocenić pod kątem występowania ciężkiej sepsy i wdrożyć odpowiednie postępowanie.

\subsection{Leczenie zachowawcze}

\begin{enumerate}
    \item Ścisła dieta.

\item \textbf{Nawadnianie chorego:} wlew i.v. 0,9\% NaCl.

\item \textbf{Leki przeciwbólowe i rozkurczowe:} jak w kolce żółciowej.

\item \textbf{Leczenie empiryczne antybiotykiem o szerokim spektrum działania:} aktywne wobec bakterii Gram-ujemnych i beztlenowych, np. cyprofloksacyna z metronidazolem; stężenie terapeutyczne w surowicy i w żółci osiągają także np. cefalosporyny II i III generacji, imipenem, aminoglikozydy, ale tylko cyprofloksacyna zachowuje stężenie lecznicze w żółci u chorych z niedrożnością dróg żółciowych.
\end{enumerate}

\subsection{Leczenie inwazyjne}

\begin{enumerate}
    \item ECPW ze sfinkterotomią endoskopową z usunięciem złogów i/lub  protezowaniem dróg żółciowych umożliwiającym prawidłowy drenaż żółci – metoda z wyboru. 
    Czas wykonania ECPW zależy od stopnia ciężkości ostrego zapalenia dróg żółciowych (ocena → patrz wyżej): ciężkie – możliwie najszybciej, w ciągu 12 h u chorego ze wstrząsem septycznym; umiarkowane – w ciągu 48–72 h; łagodne – zabieg planowy.

    \item Drenaż przezskórny pod kontrolą USG lub TK – rozważyć, gdy nie można wykonać ECPW lub nie udało się udrożnić dróg żółciowych.
\end{enumerate}

\chapter{Pierwotne stwardniające zapalenie dróg żółciowych (PSC)\index{PSC}}

Podobieństwo obrazu klinicznego do pierwotnego zapalenia dróg żółciowych.
\href{https://journals.viamedica.pl/gastroenterologia_kliniczna/article/viewFile/29305/24057}{Tutaj znajdziemy bardzo dobry artykuł: \textbf{Pierwotne stwardniające zapalenie dróg żółciowych — rozpoznawanie i leczenie}.}


Pierwotne stwardniające zapalenie dróg żółciowych (PSC) jest przewlekłą chorobą wątroby, w której dochodzi do zastoju żółci w drogach żółciowych na skutek postępującego włóknienia, niszczenia i zwężenia wewnątrzwątrobowych i zewnątrzwątrobowych dróg żółciowych. 
W chwili obecnej przyczyna tej choroby nie jest znana. 

Podejrzewa się, że u podstaw PSC leżą czynniki autoimmunologiczne (czyli mechanizmy powodujące działanie układu odpornościowego przeciw własnym tkankom). 
Prowadzi ona zwykle do marskości i niewydolności wątroby oraz do związanych z nimi powikłań.

Pierwotne stwardniające zapalenie dróg żółciowych (PSC, primary sclerosing cholangitis)  jest  przewlekłą,  cholestatyczną  chorobą wątroby charakteryzującą się naciekami zapalnymi  oraz  postępującym  zwłóknieniem  dróg  żółciowych.  
Choroba  może obejmować całe drogi żółciowe albo jedynie ich odcinki. 
W rzadkich przypadkach proces chorobowy  jest  ograniczony  do  przewodzików żółciowych, to znaczy do początkowych,międzyzrazikowych  odcinków  dróg  żółciowych.

Następstwem zmian w drogach żółciowych jest  cholestaza,  która  najczęściej  klinicznie manifestuje się świądem, a w badaniach laboratoryjnych zwiększoną aktywnością „enzymów cholestatycznych”,  jak  fosfataza  alkaliczna i  GGTP.  

\section{Obraz kliniczny}

W  obrazie  klinicznym  oprócz świądu  często  obserwuje  się  bóle  w  prawym podżebrzu,  żółtaczkę,  a  także  epizody  zapalenia  dróg  żółciowych.  
Te  ostatnie  występują szczególnie często, gdy zmianami patologicznymi objęte są duże przewody żółciowe, przez co rozumie  się  końcowe  odcinki  prawego  i  lewego przewodu wątrobowego i zewnątrzwątrobowe drogi żółciowe, to jest przewód wątrobowy wspólny  i  przewód  żółciowy  wspólny.
U 15–44\% chorych PSC przez długi czas przebiega bezobjawowo i wówczas diagnozę stawia się, szukając przyczyny przypadkowo wykrytego podwyższenia aktywności enzymów wskazujących na cholestazę (ALP, GGTP)

Etiopatogeneza  PSC  nie  jest  poznana i wydaje się, że ma bardzo złożony charakter. 
Przypuszcza się, że istotną rolę w powstaniu tej choroby odgrywają mechanizmy autoimmunologiczne, ale hipoteza ta nie jest w pełni potwierdzona.  
Przemawiają  za  nią  częste  współistnienie z PSC chorób o udowodnionej autoimmunologicznej  etiopatogenezie  oraz  obecność  różnych  autoprzeciwciał.  

Jednak  jednocześnie  nieskuteczność  leczenia  immunosupresyjnego i rzadsze występowanie u kobiet są argumentami  przeciw  autoimmunologicznej etiopatogenezie  PSC.
Pierwotne  stwardniające  zapalenie  dróg żółciowych jest chorobą rzadką. 
Roczna zapadalność na świecie waha się od 0,9 do 1,3 przypadków na 100 000 mieszkańców, a chorobowość ocenia się na 8,5–13/100 000. 
Najczęściej PSC  rozpoczyna  się  w  trzeciej  lub  czwartej dekadzie życia, ale może występować u dzieci i u osób w podeszłym wieku. 

W przeciwieństwie  do  innych  autoimmunologicznych chorób wątroby, jak pierwotna żółciowa marskość i autoimmunologiczne zapalenie, które częściej  występują  u  kobiet,  PSC  występuje głównie u mężczyzn (70–75\%).
Z  PSC  bardzo  często  współistnieją  nieswoiste  zapalenia  jelit,  głównie  wrzodziejące zapalenie jelita grubego (WZJG). 
Dotyczy to szczególnie chorych w Europie i Ameryce Północnej, wśród których aż u 63–90\% występuje nieswoiste zapalenie jelita grubego.
Typowy pacjent z PSC to mężczyzna młody lub w średnim wieku, z nieswoistą chorobą zapalną jelit (najczęściej wrzodziejącym zapaleniem), z objawami i laboratoryjnymi wykładnikami cholestazy. 
Jednak historia naturalna  PSC  wykazuje  dużą  zmienność.  
U  wielu osób choroba ujawnia się późno, w zaawansowanej fazie, po dość długim okresie bezobjawowym. 
U innych początek PSC jest burzliwy(zapalenie dróg żółciowych, żółtaczka), a choroba szybko postępuje do stadium, w którym konieczne jest pilne przeszczepienie wątroby.

Postępujące  i  nieodwracalne  uszkodzenie dróg żółciowych prowadzi do marskości wątroby  i  jej  powikłań.  

Średni  czas  przeżycia  chorych z PSC na podstawie analizy historii naturalnej  w  różnych  regionach  świata  wynosi 12–17 lat, a 10-letnie przeżycie w badaniach dużej grupy chorych w Stanach Zjednoczonych wynosiło 65\%. 
Częstym i bardzo niepomyślnym zejściem PSC jest rak wywodzący się z komórek przewodów żółciowych (cholangiocarcinoma).  
Takie  powikłanie  występu-je u 6–20\% chorych. 

Wydaje się, że w przypadku tego nowotworu onkogeneza różni się od przemiany nowotworowej prowadzącej  do  raka  wątrobowokomórkowego,w której zwykle obserwuje się sekwencję: przewlekłe zapalenie wątroby — marskość — rak.

U  chorych  z  PSC  cholangiocarcinoma  często powstaje we wczesnych okresach choroby, gdy w  wątrobie  nie  ma  jeszcze  przebudowy  marskiej.

Europejskie Towarzystwo do Badań nad Wątrobą (EASL, European Association for the Study of the Liver) w 2009 roku, a Amerykańskie Towarzystwo do Badań Chorób Wątroby(AASLD, American  Association  for  the  Study of  Liver  Disease)  w  2010  roku  opublikowały opracowane przez grupy ekspertów zaleceniadotyczące  postępowania  w  cholestatycznych chorobach wątroby, w tym w PSC (EASL) i w samym PSC (AASLD). 
Poniżej przedstawione zasady rozpoznawania i leczenia PSC oparte są na tych zaleceniach EASL-AASLD.

\vspace{0.5cm}

Rozpoznanie  typowego  przypadku PSC następuje,  gdy  spełnione  są  wszystkie  poniższe warunki:

\begin{itemize}
    \item zwiększona  aktywność  we  krwi  enzymów wskazujących  na  cholestazę;
    \item  obecne nieprawidłowości w drogach żółciowych w badaniach obrazowych [cholangiografia  rezonansu  magnetycznego  (MRCP magnetic  resonance  cholangiopancreatography) lub endoskopowa cholangiopankreatografia wsteczna (ECPW);
    \item  wykluczenie przyczyn wtórnego zwężającego zapalenia dróg żółciowych i innych chorób przebiegających z cholestazą.
\end{itemize}

Objawy podmiotowe i przedmiotowe nie mogą być podstawą rozpoznania PSC, ponieważ  u  15–44\%  pacjentów (a według innych źródeł — nawet u 15–50\% pacjentów) przez długi czas choroba jest bezobjawowa. 

Jeśli obecne są objawy, to na ogół są one niecharakterystyczne i sugerują jedynie patologię wątroby lub dróg żółciowych, a nie upoważniają do rozpoznania PSC.

Najczęściej spotykaną nieprawidłowością w badaniach laboratoryjnych u chorych z PSC jest  podwyższona  aktywność  ALP.  
Bardzo rzadko, szczególnie w początkowych okresach choroby,  aktywność  tego  enzymu  może  być prawidłowa, co opóźnia rozpoznanie. 
Wskaźnikiem wzmacniającym wartość diagnostyczną ALP  w  rozpoznaniu  cholestazy  jest GGTP.
Podwyższona  aktywność  obu  tych  enzymów wskazuje na patologię dróg żółciowych (cholestazę), ale nie rozstrzyga o rozpoznaniu PSC.

Natomiast podwyższona aktywność ALP przy prawidłowej  GGTP  lub  odwrotnie  —  zwiększona  aktywność  GGTP  przy  równocześnie prawidłowych  wartościach  ALP  —  mogą  występować w stanach niezwiązanych z chorobami  dróg  żółciowych  lub  wątroby.  
Szczególnej uwagi  co  do  obecności  PSC wymagają  mężczyźni w  młodym  lub  średnim  wieku  z  wrzodziejącym zapaleniem jelita grubego (WZJG) lub z chorobą  Crohna  mający  niecharakterystyczne „objawy  wątrobowe”  i/lub  niewielkie  nieprawidłowości  laboratoryjnych  „badań  wątrobowych”.  

Aminotransferazy  [aminotransferaza asparaginianowa (AST, aspartate  aminotransferase), aminotransferaza alaninowa (ALT, alanine aminotransferase)]  są  u  chorych  z  PSC zazwyczaj nieznacznie podwyższone i nie mają istotnej wartości diagnostycznej. 
Również stężenie  bilirubiny  w  surowicy  krwi  przez  długi czas choroby może być prawidłowe.

W  PSC  obserwuje  się  obecność  w  surowicy krwi różnych autoprzeciwciał. 

Najczęściej(u  26–94\%  chorych)  występują  przeciwciała przeciw  cytoplazmie  granulocytów  obojętnochłonnych  typu  okołojądrowego  (pANCA, perinuclear-antineutrophil cytoplasmic antibodies). 
Nierzadko obserwuje się również przeciwciała  przeciwjądrowe  (ANA,  antinuclear antibodies) i przeciw mięśniom gładkim (SMA, smooth  muscle  antibodies),  odpowiednio:u 8–77\% i u 0–83\% chorych z PSC. 

Jednak ze względu na bardzo niską swoistość przydatność oznaczania autoprzeciwciał w diagnostyce PSC jest znikoma. 
Mogą być niekiedy pomocne w przypadkach trudności w rozpoznawaniu  i  różnicowaniu  zespołów  nakładania (współistnienie PSC z innymi autoimmunologicznymi chorobami wątroby)

\textbf{Badania obrazowe:} odstawą rozpoznania PSC są nieprawidłowości  dróg  żółciowych  stwierdzane  w  badaniach  obrazowych.  
Ultrasonografia,  która jest najtańszą i powszechnie dostępną metodą obrazowania narządów jamy brzusznej, nie jest przydatna w diagnostyce PSC, ponieważ za jej pomocą  rzadko  udaje  się  uwidocznić  zmiany charakterystyczne dla tej choroby. 
Przez wiele  lat  metodą  referencyjną  rozpoznania  PSC było  ECPW.  
Badanie  to  bardzo  dobrze obrazuje nieprawidłowości występujące w zewnątrz- i wewnątrzwątrobowych drogach żółciowych, do których należą: nierówności ścian, odcinkowe  zwężenia  i  poszerzenia  światła przewodów przypominające obraz korali („be-aded” pattern). 
Zmiany mogą dotyczyć także  pęcherzyka  żółciowego  (pogrubienie i nierówność ścian) i przewodu pęcherzykowego. 

Endoskopowa  cholangiopankreatografia wsteczna jest badaniem inwazyjnym i wiąże się z ryzykiem powikłań. Najczęściej jest to zapalenie  dróg  żółciowych  lub  ostre  zapalenie trzustki. 
U chorych z PSC ryzyko zapalenia dróg żółciowych po ECPW jest 20-krotnie większe  niż  u  chorych,  u  których  ten  zabieg wykonuje się z innych wskazań. 
Ponadto, w  przypadkach  znacznych przewężeń  w  drogach żółciowych badanie ECPW może niedostatecznie  uwidoczniać  całe  drogi  żółciowe, ponieważ zwężenia uniemożliwiają zakontrastowanie ich proksymalnych odcinków.

Wobec coraz  częstszych  opinii  o  niedoskonałości ECPW i ryzyku związanym z tym badaniem od kilku  lat  coraz  powszechniejszy  staje  się  pogląd, że metodą referencyjną w obrazowaniu zmian  patologicznych  w  PSC  jest  MRCP.  
To nieinwazyjne i niegrożące powikłaniami badanie bardzo dobrze uwidacznia typowe dla PSC nieprawidłowości,  zarówno  w  wewnątrz-,  jaki w zewnątrzwątrobowych drogach żółciowych. 

W MRCP możliwe jest uwidocznienie  odcinków  dróg  żółciowych  proksymalnie znacznych  zwężeń, często  niewidoczne w ECPW, a także zobrazowanie mało zaawansowanych  zmian  w  drogach  żółciowych.
Czułość i swoistość MRCP w wykrywaniu PSC jest porównywalna z czułością i swoistością  ECPW i wynosi  odpowiednio:  83–100\% i 92–100\%. 

Wszystkie wymienione zalety MRCP oraz niższe koszty tego badania w porównaniu z kosztami ECPW sprawiają, że w  ostatnich  latach  MRCP  coraz  częściej  stawiano na równi z ECPW pod względem przydatności w rozpoznawaniu PSC. 
Natomiast najnowsze zalecenia AASLD z 2010 roku idą jeszcze  dalej  i  w  algorytmie  diagnostycznym PSC  MRCP  uznano  za  podstawowe  badanie obrazowe.  
Endoskopowa  cholangiopankreatografia wsteczna zarezerwowana jest dla szczególnych postaci PSC, jakimi są „dominujące”  zwężenia  (dominant  stenoses), gdy wskazane jest leczenie endoskopowe. Jako„dominujące”  zwężenia  określa  się  stany znacznego zwężenia przewodów wątrobowych w pobliżu ich połączenia w przewód wątrobowy  wspólny  albo  istotne  zwężenie  przewodu żółciowego wspólnego.

Zmiany te powodują znaczne upośledzenie odpływu żółci, czego następstwem są objawy ciężkiej cholestazy(żółtaczka,  świąd)  i  nawracających  zakażeń.W tych przypadkach szczególnie często dochodzi  do  tworzenia  się  złogów  nad  zwężeniami i  szybko  postępującego  uszkodzenia  wątroby prowadzącego do marskości. 
Poza tym wskazania  do  ECPW  mogą  być  nadal  rozważane w przypadkach wątpliwego obrazu MRCP albo jeśli  istnieje  podejrzenie  raka  i  planuje  się uzyskać  materiał  do  badania  cytologicznego.

U  niektórych  chorych  z  objawami  klinicznymi i nieprawidłowościami w badaniach laboratoryjnych  wskazującymi  na  PSC  badania  obrazowe  nie  ujawniają  zmian  w  wewnątrz-  ani zewnątrzwątrobowych  drogach  żółciowych.
U tych pacjentów obecna może być szczególna i  trudna  do  rozpoznania  postać  PSC,  w  której zmiany są zlokalizowane wyłącznie w początkowych  odcinkach  dróg  żółciowych,  to  jest  w  zakresie przewodzików międzyzrazikowych (small duct  PSC).  

Rozstrzygające  w  tych  przypadkach jest badanie histopatologiczne punktatu wątroby. 
Biopsja wątroby nie jest konieczna w każdym przypadku PSC. 
Wskazaniem do niej jest wspomniane podejrzenie wewnątrzwątrobowego (small duct) PSC lub konieczność wyjaśnienia, czy z PSC współistnieją inne autoimmunologiczne  choroby  (zespoły  nakładania: PSC  i  autoimmunologiczne  zapalenie  wątroby lub PSC i pierwotna żółciowa marskość wątroby). 
Niekiedy wskazaniem jest potrzeba określenia  stopnia  zaawansowania  zmian  marskich w wątrobie.

Rozpoznanie PSC wymaga wykluczenia wszystkich  innych  przyczyn  powodujących podobne zmiany w drogach żółciowych. 
Nieraz jest to zadanie bardzo trudne i może się zdarzyć, że wątpliwości nie mogą być  rozstrzygnięte.  Przykładem  może  być wtórne uszkodzenie dróg żółciowych po cholecystektomii, które morfologicznie i w badaniach  obrazowych  może  nie  różnić  się  od zmian w PSC. 

Może się również zdarzyć, że u chorego w początkowej fazie PSC, jeszcze bez  wyraźnych  nieprawidłowości  w  drogach żółciowych  w  badaniach  obrazowych,  wykonuje  się  cholecystektomię  z  powodu  współistniejącej kamicy. 
W takim przypadku trudno będzie rozstrzygnąć, czy pacjent ma PSC, czy związane z operacją wtórne stwardniające  zapalenie  dróg  żółciowych.  Nieprawidłowości  w  drogach  żółciowych  nie różniące  się od zmian w PSC mogą powstawać z wielu innych przyczyn. 
Obserwuje się je jako skutek działania  czynników  infekcyjnych,  toksycznych, immunologicznych i w różnych stanach powodujących  niedokrwienie  ścian  przewodów żółciowych, jak na przykład po zabiegach chemoembolizacji zmian w wątrobie z dojścia przez tętnicę wątrobową.

U pacjenta z rozpoznanym PSC powinno się wykonać kolonoskopię łącznie z badaniem histopatologicznym  wycinków  pobranych z  wielu  miejsc  jelita  grubego.  
Dotyczy  to wszystkich chorych z PSC, nawet jeśli nie mają objawów nieswoistego zapalenia jelit.

Program nadzoru  w  kierunku  wczesnego  wykrywania raka jelita grubego u chorych z PSC powinien się rozpoczynać od kolonoskopii w momencie rozpoznania  PSC,  ponieważ  współistnienie z PSC wrzodziejącego zapalenia jelita grubego lub choroby Crohna stanowi znaczne ryzyko rozwoju raka jelita.

Obecnie  nie  ma  skutecznego  leczenia PSC. 
Choroba jest postępująca i prowadzi do marskości  wątroby  i  zgonu  z  powodu  jej  powikłań  albo,  jak  wspomniano  wcześniej,  do rozwoju  raka  dróg  żółciowych.  

W  przypadku wystąpienia tego powikłania, nawet bez obecności  powikłań  marskości,  szanse  wyleczenia są  niewielkie.  
Ze  względu  na  wciąż  nieznaną etiopatogenezę  PSC  w  miarę  postępu  badań nad tą chorobą próbowano różnych sposobów leczenia. 
Można je podzielić na: leczenie farmakologiczne, leczenie endoskopowe i leczenie  chirurgiczne.

Poszukiwania optymalnego leczenia farmakologicznego PSC trwają od końca lat 80. ubiegłego  wieku,  kiedy  przeprowadzono pierwszą  formalną  i  niestety  zakończoną  negatywnymi  wnioskami  próbę  zastosowania D-penicylaminy. 
Wybór kolejnych leków wiązał  się  z  domniemanymi  mechanizmami patogenetycznymi PSC. 
Wśród licznych leków były preparaty przeciwzapalne lub o potencjalnym  działaniu  przeciwzwłóknieniowym,  leki zwiększające cholerezę i leki immunosupresyjne. 
Wszystkie z proponowanych sposobów  okazały  się  nieskuteczne.  
Glikokortykoisteroidy  znajdują  zastosowanie  jedynie w przypadkach współistnienia z PSC autoimmunologicznego  zapalenia  wątroby  (jedna z form zespołów nakładania) [8, 11].Najwięcej  nadziei  w  farmakologicznym leczeniu PSC budziły próby z kwasem ursodezoksycholowym  (UDCA,  ursodeoxycholic acid).  

Zakładano,  że  jego  wielokierunkowe działanie  —  zwiększanie  produkcji  i  szybkości wypływu żółci z wątroby, działanie cytoprotekcyjne,  immunomodulujące  i  antyapoptotyczne — będzie korzystne w przewlekłej chorobie  cholestatycznej,  jaką  jest  PSC.

W  wielu  pracach  dotyczących  prowadzonych w  ostatniej  dekadzie  badań  z  zastosowaniem standardowych  dawek  UDCA  (10–20  mg/kg//dobę)  obserwowano  zwykle  przejściową  poprawę w zakresie niektórych parametrów biochemicznych, zmniejszenie objawów, a nawet(co  udokumentowano  w  niewielu  pracach) poprawę  obrazu  histologicznego  wątroby.

Z zestawienia wyników wszystkich badań nadUDCA  opracowanego  przez  Tischendorfa i wsp. wyraźnie wynika, że w żadnym nie wykazano korzystnego wpływu tego leku na przeżycie  chorych  z  PSC.  
Rezultaty  kolejnych badań  klinicznych  z  użyciem  znacznie  większych,  ale  jeszcze  nietoksycznych  dawek UDCA (20–30 mg/kg/dobę) były bardziej optymistyczne. 

W wieloośrodkowym europejskim badaniu przeprowadzonym w ponad 100-osobowej grupie chorych z PSC leczonych przez 5 lat dużymi dawkami UDCA, opublikowanym w  2005  roku  wykazano  poprawę  w  zakresie parametrów laboratoryjnych, klinicznych i histologicznych,  a  także  zasygnalizowało  korzystny trend dotyczący przeżycia tych chorych.  
Jednak  te  zachęcające  wstępne  wyniki przyjmowano  z  ostrożnością,  co  znalazło  wyraz w nieuwzględnieniu w zaleceniach EASL z  2009  roku  UDCA  jako  leku  zalecanego w  PSC.  

Już  po  opublikowaniu  zaleceń EASL ukazała się kolejna praca przedstawiająca wyniki wieloletnich, wieloośrodkowych badań nad skutecznością dużych dawek UDCA (28–30  mg/kg/dobę),  w  której  wykazano,  że mimo  poprawy  parametrów  laboratoryjnych końcowy  efekt  leczenia  był  gorszy  niż  w  grupie kontrolnej. 

W grupie leczonej dużymi dawkami UDCA u większej liczby chorych rozwinęła się marskość wątroby, żylaki przełyku, rak dróg żółciowych i było więcej wskazań do przeszczepienia  wątroby.  
W  wyniku  tych  do-niesień  AASLD  w  swoich  najnowszych  zaleceniach jednoznacznie wypowiada się przeciw stosowaniu UDCA w PSC.

Natomiast istnieją rozbieżności między stanowiskiem towarzystwa  europejskiego  (EASL)  i  amerykańskiego (AASLD) w kwestii stosowania UDCA w prewencji raka jelita grubego. Europejskie Towarzystwo do Badań nad Wątrobą dopuszcza, a AASLD nie zaleca takiego postępowania

Jeszcze do niedawna leczenie endoskopowe było powszechnie stosowane u chorych z PSC. 
Zakładano, że mimo iż choroba obejmuje często części układu żółciowego niedostępne rozszerzaniu endoskopowemu, to odbarczenie  dystalnych  odcinków  przewodów żółciowych (nacięcie brodawki Vatera, protezowanie dróg żółciowych) poprawi drenaż żółci z całej wątroby i korzystnie wpłynie na przebieg PSC. Jednak w praktyce nie uzyskiwano zamierzonych efektów w oczekiwanym wymiarze.  

Częstość  powikłań  miejscowych i ogólnych była podobna, a nawet większa, niż po  zabiegowych  ECPW  przeprowadzanych z  innych  wskazań.  
Obecnie  leczenie  endoskopowe  polegające  na  ECPW  i  rozszerzaniu przewodów żółciowych za pomocą balonów albo na wprowadzeniu protez jest zarezerwowane  dla  szczególnych  form  PSC,  tojest  przypadków  z  „dominującym”  zwężeniem. 
Stanowią one do 35\% wszystkich  przypadków  PSC.  

Zabiegów endoskopowych  nie  można  powtarzać  wielokrotnie, ponieważ nawet w tych specjalnie kwalifikowanych  przypadkach  po  szeregu procedur  endoskopowych  znacznie  wzrasta ryzyko  groźnych  powikłań  (zakażenia,  perforacje)  albo  trwałych  niekorzystnych  następstw (wtórne zwężenia).

Leczenie operacyjne chorych z PSC z dominującym  zajęciem  zewnątrzwątrobowych dróg żółciowych, polegające na resekcji, zespoleniach omijających znaczne zwężenia lub śródoperacyjnym protezowaniu, bywa stosowane, ale  jest  trudne,  ryzykowne  i  wymaga  dużego doświadczenia zespołu chirurgicznego. 
Chorzy muszą  być  starannie  kwalifikowani,  moment i warunki operacji są trudne do ustalenia, a wyniki leczenia nie są trwałe i zadowalające.

Obecnie  leczenie  chirurgiczne  PSC  rezerwuje się dla szczególnych przypadków nie nadających się do innego leczenia i spełniających ostre kryteria  kwalifikacyjne.  
Natomiast  ortotopowy przeszczep wątroby ma trwałe miejsce w skromnym  arsenale  metod  leczenia  PSC.  

Pacjenci z  PSC  mogą  być  kierowani  na  przeszczep  wątroby z dwojakich wskazań. 
Pierwsza grupa to zaawansowana  marskość  wątroby  nie rokująca poprawy po klasycznym leczeniu albo zagrażające życiu powikłania marskości. 
Tryb postępowania kwalifikacyjnego jest u tych chorych taki sam,  jak  w  przypadkach  marskości  wątroby o innej etiologii. 

Druga grupa wskazań do przeszczepu wątroby u chorych z PSC to podejrzenie  raka  dróg  żółciowych.  
Jest  to  sytuacja o  wiele  trudniejsza  w  porównaniu  ze  wskazaniami  wynikającymi  z  marskości  wątroby,  ponieważ dotychczas nie udało się opracować skutecznych metod nadzoru pacjentów z PSC pod kątem  wczesnego  wykrywania  cholangiocarcinoma i odpowiednio wczesnej kwalifikacji chorych  do  przeszczepu  z  tych  wskazań. 

Badania obrazowe  jak  ECPW  lub  MRCP,  bardzo  skuteczne  w  rozpoznawaniu  PSC,  zawodzą  jako czuła  metoda  wykrywania  raka.  
Niską  czułośćma  również  systematyczne  monitorowanie u chorych z PSC stężeń w surowicy krwi markerów  nowotworowych  CA-19-9,  CEA  (carcinoembryonic antigen) lub ich kombinacji. 
Także pozytronowa tomografia emisyjna (PET, positron emission tomography) nie jest dostatecznie  czułą  metodą  wczesnego  rozpoznawania raka dróg żółciowych wikłającego PSC. 

Tak więc, wobec braku dobrych metod wspomagających  ustalenie  optymalnego  momentu  przeszczepu wątroby u chorych z prawdopodobieństwem raka dróg żółciowych w przebiegu PSC, lekarz  prowadzący  takich  chorych  musi  z  jednej  strony  wykazywać  dużą  „czujność  onkologiczną”,  ale  z  drugiej  strony,  powinien  wykorzystywać wspomniane metody z odpowiednio ograniczonym  zaufaniem.  
W  Europie  pewne rozpoznanie  raka  dróg  żółciowych  u  chorego z PSC jest przeciwwskazaniem do przeszczepu wątroby. 

Natomiast w Stanach Zjednoczonych w sytuacji rozpoznania cholangiocarcinoma najpierw zaleca się rozważenie resekcji chirurgicznej, a tylko w szczególnych przypadkach dopuszcza  się  próbę  przeszczepienia  wątroby poprzedzoną  leczeniem  neoadjuwantowym. 
Wyniki przeszczepiania wątroby u chorych z PSC (bez przypadków raka wykrytego przed zakwalifikowaniem  do  przeszczepu)  są  dobre.
Pięcioletnie przeżycie wynosi 83–89\%.

Istotnym problemem  mogą  być  powikłania  pooperacyjne  wynikające  z  dużych  zmian  anatomicznych w  zewnątrzwątrobowych  drogach  żółciowych biorcy. 
Drugim ważnym problemem jest nawrót PSC  w  przeszczepionym  narządzie.  
Nawroty choroby,  z  powodu  której  dokonywano  przeszczepu,  obserwuje  się  nie  tylko  w  PSC,  ale w  tej  chorobie  odsetek  nawrotów  jest  wysoki i wynosi 9–20\%. 

Obecne metody rozpoznawania i leczenia PSC nie są optymalne. 
Wciąż wymagają zmian i wprowadzania nowych, skuteczniejszych rozwiązań. 

\vspace{0.5cm}

Najważniejsze punkty  z  zaleceń EASL i AASLD\index{wytyczne EASL-AASLD}, które należy stosować w  praktyce:

\begin{enumerate}
    \item Podstawą rozpoznania PSC są cechy kliniczne  i  laboratoryjne  cholestazy  oraz zmiany w obrazie dróg żółciowych uwidocznione  za pomocą  MRCP.  
    
    Badanie ECPW,  jako  bardziej  inwazyjne,  ma obecnie  mniejsze  zastosowanie  w  diagnostyce  PSC.
    Z  chwilą  rozpoznania PSC u każdego chorego należy wykonać kolonoskopię łącznie z badaniem histopatologicznym wycinków błony śluzowej jelita  grubego.
    
    \item   Chorzy z PSC wymagają starannego i regularnego  nadzoru  (badanie  kliniczne, laboratoryjne i badania obrazowe) w celu jak  najwcześniejszego  wykrycia  zmian wskazujących na możliwość rozwoju raka dróg żółciowych. 
    
    Obecnie nie ma jednolitych  i  powszechnie  akceptowanych  zasad  takiego  nadzoru.  
    W  przypadkach współistnienia colitis ulcerosa lub choroby  Crohna  program nadzoru  chorego z PSC powinien obejmować również kolonoskopię  z  badaniem  histologicznym.
    Kolonoskopię powinno się wykonywać \textbf{co 12 miesięcy}.
    
    \item   Nie  ma  wg obecnych wytycznych podstaw  do  stosowania UDCA u chorych z PSC.
    
    \item   UDCA można stosować w  ramach  chemoprewencji  raka  jelita  grubego u chorych z PSC, ale to postępowanie nie  jest  powszechnie  zaakceptowane.
    
    \item   Glikokortykosteroidy  nie  są  zalecane w PSC. 
    Ich zastosowanie może być rozważane jedynie w przypadkach zespołów nakładania.
    
    \item Pierwotne  stwardniające  zapalenie  dróg żółciowych (PSC)  z  „dominującym”  zwężeniem dróg żółciowych jest wskazaniem do endoskopowego rozszerzania zwężenia. 
    
    W przypadku  nieskuteczności  leczenia  endoskopowego należy rozważać leczenie operacyjne lub przeszczepienie wątroby.
    
    \item   Przeszczepienie  wątroby  jest  leczeniem z wyboru w zaawansowanych stadiach PSC i w przypadkach podejrzenia raka dróg żółciowych
\end{enumerate}


Choroba może rozwinąć się w każdym wieku, ale najczęściej dotyczy mężczyzn w wieku od 30 do 50 roku życia (mężczyźni chorują dwukrotnie częściej niż kobiety). 

U około 15\% chorych na PSC dochodzi do rozwoju raka dróg żółciowych. 
Ponadto u 70\% pacjentów z PSC stwierdza się także nieswoiste choroby zapalne jelit (wrzodziejące zapalenie jelita grubego i, rzadziej, chorobę Leśniowskiego i Crohna). 

Choroba przebiega w sposób podstępny (objawy pojawiają się późno) i postępujący. 
Pierwsze objawy choroby to zwykle osłabienie, utrata masy ciała, nadmierna senność i świąd skóry, który jest szczególnie uciążliwy w nocy. 
U przeważającej liczby pacjentów świąd poprzedza wystąpienie żółtaczki, która jest późnym objawem choroby. Ponadto chorzy zgłaszają uczucie dyskomfortu lub ból w prawym podżebrzu. 
Nawracający ból w tej okolicy z towarzyszącą gorączką i żółtaczką najczęściej świadczy o zapaleniu dróg żółciowych.

\section{Różnicowanie}

Różnicowanie PBC  z PSC przedstawione w tabeli \ref{tab:PBC_vs_PSC} a stronie \pageref{tab:PBC_vs_PSC}.

\begin{sidewaystable}
    \centering
    \caption{Pierwotne zapalenie dróg żółciowych, inaczej pierwotna marskość wątroby (PBC) i pierwotne stwardniające zapalenie dróg żółciowych (PSC)}
    \label{tab:PBC_vs_PSC}
    \begin{tabular}{|p{5cm}|p{9cm}|p{9cm}|}
    \hline 
    \textbf{Istota choroby} & zapalenie międzyzrazikowych dróg żółciowych & przewlekłe destrukcyjne zapalenie zewnątrz- i/lub wewnątrzwątrobowych dróg żółciowych z powstawaniem zwężeń, co prowadzi do marskości i schyłkowej niewydolności wątroby \\
    \hline   
    \textbf{Chorobowość} & zróżnicowana regularnie: 1.91 do 40.2/100 tysięcy mieszkańców  & w krajach rozwiniętych 8.5-13/100 tysięcy mieszkańców \\
    \hline
    \textbf{Płeć: 
        kobiety / mężczyźni} &  & \\
    \hline
    \textbf{Etiopatogeneza} & predyspozycja genetyczna, czynniki środowiskowe: palenie tytoniu, infekcje dróg moczowych, powszechnie występujące bakterie (\textit{Novosphingobium aromaticivorans}), ksenobiotyki (kwas oktynowy - dodatek smakowy), współistnienie IBD (rzadkie, rzadsze niż w przypadku PSC).    & predyspozycja genetyczna, czynniki środowiskowe, translokacja bakterii jelitowych i ich komponentów (lipopolisacharydy) do krążenia wrotnego, aktywacja komórek macierzystych wokół przewodów żółciowych, współistnienie IBD (głównie WZJG) jest częstsze niż w PBC bo wynosi 20-80\%.\\
    \hline
    \textbf{Objawy} & zmęczenie, świąd skóry, objawy osteodystrofii (objawy późne: żółtaczka, niewyrównana marskość) & pobolewania w prawym podżebrzu, zmęczenie, świąd skóry, ubytek masy ciała, żółtaczka, powiększenie wątroby i śledziony w późnym okresie \\
    \hline
    \textbf{Badania laboratoryjne} & cechy cholestazy, cholesterol $\uparrow$, IgM $\uparrow$, AMA(+) & cechy cholestazy, cholesterol $\uparrow$ i IgM często $\uparrow$, AMA(-)\\
    \hline
    \textbf{Badania obrazujące} & USG, TK, RM norma (lub cechy marskości w późnym okresie choroby) & MRCP, ECPW:naprzemienne zwężenia i poszerzenia dróg żółciowych (,,sznur pereł", USG bez znaczenia dla rozpoznania zwężeń \\
    \hline
    \end{tabular}
\end{sidewaystable}

\chapter{Pierwotne stwardniające zapalenie dróg żółciowych (PSC)\index{PSC} wg Szczeklika}

\section{\DiEtp}

Przewlekła, cholestatyczna choroba wątroby o złożonej etiologii, prowadząca do uszkodzenia wewnątrzwątrobowych i zewnątrzwątrobowych dróg żółciowych.


\section{\ObriP}

U 15–45\% chorych przebieg bezobjawowy. 

\textbf{Objawy:} przewlekłe zmęczenie, świąd skóry, utrata masy ciała, objawy nawracającego zapalenia dróg żółciowych (epizody żółtaczki, gorączki i bólu w okolicy podżebrowej prawej).

W badaniu przedmiotowym zażółcenie skóry i błon śluzowych, przeczosy. 
W stadium zaawansowanym mogą wystąpić objawy marskości wątroby i jej powikłań. 
U ~70\% chorych współistnieje wrzodziejące zapalenie jelita grubego (WZJG) lub – rzadziej – choroba Leśniowskiego i Crohna (ChLC); także zapalenie trzustki (10–25\%), cukrzyca (5–15\%), inne choroby autoimmunologiczne. 
W ciągu 10–15 lat od rozpoznania $>$50\% chorych wymaga przeszczepienia wątroby.

\section{Rozpoznanie}

\subsection{Badania pomocnicze}

\begin{enumerate}
    \item \textbf{Badania laboratoryjne:}
    
    \begin{enumerate}
        \item badania biochemiczne krwi – zwiększona aktywność ALP i/lub GGT (zwykle 2–10 × ggn) oraz AST i ALT (zwykle 2–4 × ggn) w surowicy, zwiększone stężenie bilirubiny w surowicy (u ~60\% chorych prawidłowe w chwili rozpoznania), hipergammaglobulinemia (zwiększenie stężenia IgG i IgM u 45–80\% chorych)

    \item badania immunologiczne – autoprzeciwciała ANA i SMA (u 20–50\%), pANCA (u 30–80\%).
\end{enumerate}

    \item \textbf{MRCP\index{MRCP}:} złoty standard diagnostyczny; umożliwia wykazanie charakterystycznych zwężeń z następczym poszerzeniem dróg żółciowych (mogą wystąpić zarówno w odcinku wewnątrzwątrobowym, jak i zewnątrzwątrobowym).
    
    \item \textbf{ECPW:} badanie inwazyjne, powinno się je wykonywać wtedy, kiedy MRCP nie pozwala potwierdzić rozpoznania oraz w sytuacji klinicznej wymagającej bezpośredniej interwencji leczniczej (sfinkterotomia, protezowanie dróg żółciowych) i pobrania materiału do badania cytologicznego w razie podejrzenia raka dróg żółciowych.
    
    \item \textbf{Biopsja wątroby:} wskazana w razie podejrzenia PSC z zajęciem drobnych przewodzików, którego nie można potwierdzić w badaniach obrazowych, a także przy współistnieniu autoimmunologicznego zapalenia wątroby, gdy rozpoznanie na podstawie MRCP lub ECPW nie jest pewne oraz gdy istnieje kliniczne podejrzenie marskości wątroby (potwierdzenie zaawansowanego włóknienia ma istotne implikacje kliniczne).
\end{enumerate}

\subsection{Kryteria rozpoznania}

Typowy obraz dróg żółciowych w ECPW lub MRCP. 
Jeśli obraz w tych badaniach jest prawidłowy a obraz kliniczny wskazuje na PSC → badanie histologiczne wycinka wątroby (jeśli obraz histologiczny jest typowy dla PSC, rozpoznaje się wariant PSC z zajęciem tylko drobnych dróg żółciowych).

\subsection{Rozpoznanie różnicowe}

Stwardniające zapalenie dróg żółciowych związane z IgG4 (jedna z manifestacji choroby związanej z IgG4), inne przewlekłe choroby wątroby przebiegające z cholestazą (np. pierwotne zapalenie dróg żółciowych, zespół zanikających dróg żółciowych, eozynofilowe zapalenie dróg żółciowych), wtórne stwardniające zapalenie dróg żółciowych (np. po jatrogennym uszkodzeniu dróg żółciowych lub naczyń odpowiedzialnych za ich ukrwienie), wrodzone wady dróg żółciowych (np. zespół Alagille’a\index{Alagille’a zespół} [hipoplazja wewnątrzwątrobowych dróg żółciowych], zespół Carolego\index{Carolego zespół} [torbielowatość wewnątrzwątrobowych dróg żółciowych]), cholangiopatie infekcyjne (np. u chorych na AIDS, zarażenie chińską przywrą wątrobową).

\section{Leczenie}

\subsection{Leczenie farmakologiczne}

\begin{enumerate}
    \item Kwas ursodeoksycholowy (Proursan, Ursocam, Ursofalk, Ursopol, Ursoxyn) p.o. 13–15 mg/kg/d – prawdopodobnie zmniejsza ryzyko dysplazji i być może też raka jelita grubego u chorych na PSC ze współistniejącym WZJG.

    \item Leczenie świądu

    \item Profilaktyka i leczenie osteoporozy.

    \item Ostre zapalenie dróg żółciowych wymaga szybkiej i agresywnej antybiotykoterapii (chinolon lub cefalosporyna III generacji i metronidazol i.v.) z uwagi na ryzyko wystąpienia sepsy.
\end{enumerate}

\subsection{Leczenie inwazyjne}

\begin{enumerate}
    \item \textbf{Leczenie endoskopowe:} u chorych z dominującym zwężeniem dróg żółciowych można wykonać endoskopowe rozszerzanie balonem lub protezowanie dróg żółciowych (skuteczność 60–90\%).

    \item \textbf{Przeszczepienie wątroby:} wskazania – nawracające epizody ostrego zapalenia dróg żółciowych, brak poprawy po leczeniu farmakologicznym i endoskopowym zaawansowanych zwężeń dróg żółciowych, cechy opornej na leczenie zachowawcze schyłkowej niewydolności wątroby, oporny na leczenie świąd skóry. 
    Po transplantacji 5 lat przeżywa ~80\%. 
    PSC może nawrócić w przeszczepie.
\end{enumerate}

\section{Powikłania}

\begin{enumerate}
    \item zwiększone ryzyko rozwoju nowotworów złośliwych – rak dróg żółciowych (u 10–20\%), rak wątrobowokomórkowy (ryzyko zwiększone, gdy rozwinie się marskość wątroby), rak trzustki (ryzyko 14-krotnie większe niż w populacji ogólnej), rak jelita grubego (PSC dodatkowo zwiększa ryzyko związane z WZJG), rak pęcherzyka żółciowego (u 2\%)

    \item epizody ostrego zapalenia dróg żółciowych
\end{enumerate}



\chapter{Rak pęcherzyka żółciowego}

\section{\DiEtp}

Najczęściej jest to gruczolakorak (\textit{adenocarcinoma}) wywodzący się z nabłonka błony śluzowej pęcherzyka żółciowego.

\textbf{Czynniki ryzyka:} wieloletnia kamica pęcherzyka żółciowego (zwłaszcza złogi $>$3 cm; kamienie żółciowe w pęcherzyku w ~80\% przypadków raka; raka rozpoznaje się w ~1\% pęcherzyków usuwanych z powodu kamicy), torbiele dróg żółciowych, polip pęcherzyka żółciowego $>$1 cm, zespoły rodzinnej polipowatości. 

\section{\ObriP}

Objawy niecharakterystyczne, występują zwykle późno (często rozpoznanie już w stadium nieoperacyjnym), najczęściej ból w okolicy podżebrowej prawej, tępy, promieniujący w prawą stronę do kręgosłupa i okolicy międzyłopatkowej; żółtaczka i świąd (w wyniku naciekania przez nowotwór przewodu żółciowego; pogarsza rokowanie); nudności i wymioty; utrata łaknienia, utrata masy ciała; guz wyczuwalny w prawym górnym kwadrancie brzucha.

\section{Rozpoznanie}

\subsection{Badania pomocnicze}

\begin{enumerate}
    \item \textbf{Badania laboratoryjne:} w surowicy zwiększona aktywność ALP i GGT, stężenie bilirubiny, aktywność ALT i AST (w fazie znacznego zaawansowania choroby, naciekania wątroby i zmian przerzutowych); zwiększenie stężenia CEA i CA19-9.

    \item \textbf{Badania obrazowe:} USG – może wykazać pogrubienie ściany ($>$1 cm), guz w świetle lub na zewnątrz pęcherzyka (polipy pęcherzyka o średnicy $<$1 cm rzadko są rakiem). 
    
    EUS – określenie głębokości naciekania raka oraz ocena zajęcia regionalnych węzłów chłonnych. 
    
    TK – zmiany podobne jak w USG; dodatkowo ocena topografii zmian, otaczających węzłów chłonnych i naciekania wątroby. 
    
    MRCP i ECPW – mogą ujawnić zwężenia środkowego odcinka dróg żółciowych, co wskazuje na ich naciekanie przez raka.
\end{enumerate}

\subsection{Kryteria rozpoznania}

Podejrzenie na podstawie wyników badań obrazowych; pewne rozpoznanie na podstawie badania histologicznego usuniętego pęcherzyka żółciowego.

\section{Leczenie}

\begin{enumerate}
    \item \textbf{Leczenie z intencją wyleczenia:} leczenie operacyjne (zabieg radykalny możliwy tylko we wczesnej fazie choroby) – cholecystektomia metodą otwartą (wycięcie pęcherzyka z szerokim [$\geq$2 cm] marginesem wątroby + usunięcie regionalnych węzłów chłonnych). 
    
    Jeśli niezaawansowanego miejscowo raka pęcherzyka rozpoznano przypadkowo w badaniu histologicznym przeprowadzonym po cholecystektomii laparoskopowej, nie ma uzasadnienia dla otwartej reoperacji.

    \item \textbf{Leczenie paliatywne:} drenaż dróg żółciowych metodą endoskopową, z wszczepieniem stentu, w celu zmniejszenia objawów cholestazy.
\end{enumerate}

\section{Rokowanie}

Średni czas przeżycia ~6 mies. 
Czas przeżycia po operacji zależy od stopnia zaawansowania klinicznego: od $>$90\% w stopniu I do maks. 15\% 5-letnich w stopniu IV (wg TNM).

\chapter{Rak dróg żółciowych}

\section{\DiEtp}

Najczęściej gruczolakorak (95\%) wywodzący się z nabłonka błony śluzowej dróg żółciowych. 

\textbf{Czynniki ryzyka:} pierwotne stwardniające zapalenie dróg żółciowych (PSC), wrzodziejące zapalenie jelita grubego, torbiele dróg żółciowych (w tym zespół Carolego), kamica przewodowa (zwłaszcza wtórna do przewlekłego zapalenia dróg żółciowych), infestacja przywrami Clonorchis sinensis, Opisthorchis viverrini, Opisthorchis felineus. 

\paragraph{Podział na podstawie lokalizacji:}

\begin{enumerate}
    \item rak wewnątrzwątrobowych dróg żółciowych

    \item rak zewnątrzwątrobowych dróg żółciowych – górny (guz Klatskina\index{Klatskina guz}; powyżej lub obejmujący miejsce podziału przewodu wątrobowego wspólnego na prawy i lewy przewód wątrobowy) i dolny.
\end{enumerate}

\section{\ObriP}

\textbf{Objawy:} żółtaczka cholestatyczna (jeśli rak zlokalizowany zewnątrzwątrobowo; w przypadku lokalizacji wewnątrzwątrobowej nie zawsze występuje), świąd skóry, dyskomfort i ból brzucha (zwykle w przypadku wewnątrzwątrobowej lokalizacji; stały, tępy, w okolicy podżebrowej prawej), utrata masy ciała, powiększenie wątroby, gorączka, wyczuwalny przez powłoki guz w prawym podżebrzu, powiększony, twardy, niebolesny pęcherzyk żółciowy (objaw Courvoisiera\index{Courvoisiera objaw} – w guzach zlokalizowanych poniżej połączenia przewodu pęcherzykowego z przewodem wątrobowym wspólnym).

Przebieg zwykle podstępny. 
Pojawienie się żółtaczki i świądu świadczy zwykle o znacznym zaawansowaniu raka; u większości chorych na tym etapie jest już nieresekcyjny, a czas przeżycia nie przekracza zwykle 12 mies. od chwili rozpoznania.

\section{Rozpoznanie}

\subsection{Badania pomocnicze}

\begin{enumerate}
    \item \textbf{Badania laboratoryjne:} w surowicy zwiększone stężenie bilirubiny (z przewagą sprzężonej), zwiększona aktywność ALP, GGT, ALT i AST; zwiększenie stężenia antygenów CA 19-9 (utrzymujące się po odbarczeniu żółtaczki mechanicznej) i CA 125.

    \item \textbf{Badania obrazowe:} 
    
    \begin{itemize}
        \item USG – badanie pierwszego wyboru w diagnostyce żółtaczki mechanicznej; może uwidocznić poszerzenie dróg żółciowych. 
    
    \item TK – umożliwia wykrycie zmiany ogniskowej i określenie zaawansowania nowotworu (wykrycie przerzutów do węzłów chłonnych, naciekania dużych naczyń i narządów sąsiednich). 
    
    \item MR – rozstrzygające badanie pozwalające na uwidocznienie guza. 
    
    \item MRCP – uwidocznienie zwężeń/poszerzeń dróg żółciowych dokładniej niż w ECPW, ocena rozległości guza. 
    
    \item EUS – dokładna ocena zewnątrzwątrobowych dróg żółciowych, pęcherzyka żółciowego, struktur we wnęce wątroby, regionalnych węzłów chłonnych i naczyń; biopsja guza lub węzłów chłonnych pod kontrolą EUS jest bardzo czułą metodą diagnostyczną.
    
    \item ECPW – umożliwia pobranie wycinka lub wymazu szczoteczkowego i/lub wprowadzenie protezy do dróg żółciowych poprzez miejsce zwężenia w celu poprawy odpływu żółci. 
    
    \item Cholangioskopia – umożliwia wizualizację dróg żółciowych, pobranie wymazu szczoteczkowego lub wycinka ze zmiany. 
    
    \item PET – rozpoznawanie przerzutów odległych, monitorowanie leczenia i wznowy.
    \end{itemize}

    \item \textbf{Badanie mikroskopowe:} materiał pobrany podczas ECPW, cholangioskopii lub biopsji igłowej pod kontrolą USG, TK lub EUS.
\end{enumerate}

\subsection{Kryteria rozpoznania}

Rak głowy trzustki, brodawki Vatera, dwunastnicy lub pęcherzyka żółciowego; zwężenie dróg żółciowych (zwykle pooperacyjne); pierwotne stwardniające zapalenie dróg żółciowych; kamica dróg żółciowych; zespół Mirizziego; przerzuty do wątroby; stwardniające zapalenie dróg żółciowych związane z IgG4.

\section{Leczenie}

\begin{enumerate}
    \item \textbf{Leczenie operacyjne:} rak wewnątrzwątrobowy → hemihepatektomia; rak zewnątrzwątrobowych dróg żółciowych → zakres resekcji zależy od położenia guza względem wnęki wątroby i głowy trzustki (resekcja odcinkowa dróg żółciowych, ew. poszerzona o odpowiednie segmenty wątroby, hemihepatektomia, a w przypadku dystalnego położenia zmiany – pankreatoduodenektomia). Po operacji radykalnej w przypadku korzystnego (obwodowego) położenia guza 5 lat przeżywa do 40% chorych.

    \item \textbf{Adiuwantowa radioterapia i chemioterapia:} Adiuwantowa radioterapia i chemioterapia nie wydłużają przeżycia; w niektórych ośrodkach stosuje się radioterapię i chemioterapię (np. gemcytabinę z cisplatyną lub oksaliplatyną).

    \item \textbf{Leczenie paliatywne:} w celu zmniejszenia objawów cholestazy drenaż dróg żółciowych metodą endoskopową, z wszczepieniem stentu, rzadziej drenaż przezskórny lub operacyjny. 
    U chorych z przeciwwskazaniami do resekcji ablacja guza (np. mikrofalowa) może zahamować progresję choroby. 
    Rokowanie tym gorsze, im bardziej dowątrobowo rozwija się rak dróg zewnątrzwątrobowych.
\end{enumerate}

\chapter{Rak brodawki Vatera}

Raki brodawki większej dwunastnicy (Vatera) rozrastają się w miejscu połączenia dróg żółciowych z dwunastnicą. 
Czynnikiem ryzyka jest zespół rodzinnej polipowatości gruczolakowatej.

\section{Objawy}
Objawy jak w raku dróg żółciowych zlokalizowanym w obwodowej części zewnątrzwątrobowych dróg żółciowych, ale wcześniej pojawia się żółtaczka (stąd rozpoznanie w dość wczesnym stadium zaawansowania), może wystąpić niedrożność dwunastnicy.

\section{Rozpoznanie}
Duodenoskopia lub ECPW z pobraniem wycinków lub wymazu szczoteczkowego.

\section{Leczenie}
Pankreatoduodenektomia; we wczesnym stadium możliwa endoskopowa resekcja samej brodawki Vatera (ampulektomia). 
W przypadku zmiany nieresekcyjnej endoskopowa papillotomia i drenaż dróg żółciowych w celu złagodzeniach objawów cholestazy.

\printindex
\end{document} 

